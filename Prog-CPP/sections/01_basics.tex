\section{Basics}

\subsection{Grundsätzlich}

C++ ist sehr ähnlich zu C. Es gibt aber einige wichtige Änderungen zu C:

\begin{itemize}[itemsep=1pt, parsep=0pt]
    \item zu benutzender Compiler heisst jetzt \texttt{clang++}
    \item Nativen Bool Typ : \textbf{bool}
    \item richtiges Konstanten Schlüsselwort : \textbf{constexpr}
    \item Standartbibliotheken haben kein .h mehr beim Aufrufen
    \item C Standartbibliotheken können weiter verwendet werden, haben aber ein C im Header(stdio.h $\rightarrow$ cstdio)
    \item Char-Literale (z. B. 'A') haben in C++ den Datentyp char (in C war es int)(Brauchen ein const)
    \item Es gibt benannte Namensräume. Wichtigster Namensraum für die STL: \texttt{std}
\end{itemize}

\subsection{Hello world C++}

\lstinputlisting{code/hello_World.cpp}

\subsection{Range-based for Loop}\label{forEach}

Range based for Loops laufen automatisch ein Array ab, ohne das zuerst die Grösse ermittelt werden muss. 
Die C for loop syntax wurde etwas erweitert.
Syntax:

\lstinputlisting{code/forEach.cpp}

Siehe: Als Array Name wird ein \texttt{Pointer} auf das Array verlangt, nicht auf das erste Element!  
Ein foreach funktioniert \textbf{nicht} mit einem \textbf{Pointerpointer}. 

\subsection{Streams}

Ein Stream repräsentiert einen generischen sequentiellen Datenstrom. Z.b: ein Eingabefeld, Dateien oder Netzwerktraffic.\\
Die wichtigsten Operatoren sind:

\begin{itemize}[itemsep=1pt, parsep=0pt]
    \item \verb|<<| : Inserter $\rightarrow$  Daten einfügen
    \item \verb|>>| :  Extractor $\rightarrow$ Daten herausholen
\end{itemize}

Für Standardklassen sind diese Operatoren bereits definiert.

\subsubsection*{Standardstreams}

\begin{itemize}[itemsep=1pt, parsep=0pt]
    \item \textbf{cin} : Standard Eingabe
    \item \textbf{cout} : Standard Ausgabe
    \item \textbf{cerr} : Standard Fehlerausgabe, 
    \item \textbf{clog} : Standard Log Ausgabe, möglicherweise mit cerr gekoppelt 
\end{itemize}\vspace{0.5em}

\textbf{\texttt{cin}/\texttt{cout} Immer ganz links!}, operatoren klar und lesbar anordnen

\subsubsection{Streamformatierung}

Formatierungen der Streams können mit folgenden Schlüsselwörtern erreicht werden:


\begin{center}
    \begin{tabular}{ll}
        \rowcolor[RGB]{239,239,239} 
        \textbf{Flag} & \textbf{Wirkung}                                        \\ \hline
        boolalpha     & bool-Werte werden textuell ausgegeben                   \\
        dec           & Ausgabe erfolgt dezimal                                 \\
        fixed         & Gleitkommazahlen im Fixpunktformat                      \\
        hex           & Ausgabe erfolgt hexadezimal                             \\
        internal      & Ausgabe innerhalb Feld                                  \\
        left          & linksbündig                                             \\
        oct           & Ausgabe erfolgt oktal                                   \\
        right         & rechtsbündig                                            \\
        scientific    & Gleitkommazahl wissenschaftlich                         \\
        showbase      & Zahlenbasis wird gezeigt                                \\
        showpoint     & Dezimalpunkt wird immer ausgegeben                      \\
        showpos       & Vorzeichen bei positiven Zahlen anzeigen                \\
        skipws        & Führende Whitespaces nicht anzeigen                     \\
        unitbuf       & Leert Buffer des Outputstreams nach Schreiben           \\
        uppercase     & Alle Kleinbuchstaben in Grossbuchstaben wandeln
    \end{tabular}
\end{center}

\cor{Folgende Funktionen setzten Header \textbf{$<$iomanip$>$} voraus:}


\begin{tabular}{|l|p{4cm}|p{2cm}|}
        \hline
        \textbf{Manipulator}     & \textbf{Effekt / Funktionsweise} & \textbf{Gültigkeit} \\ 
        \hline
        \texttt{setw(n)}         & Setzt die minimale Feldbreite. Ist der Inhalt kürzer, wird mit Füllzeichen (Standard: Leerzeichen) aufgefüllt. & \textbf{Einmalig. Auch für Strings.} \\ 
        \hline
        \texttt{setprecision(n)} & \textbf{Ohne \texttt{fixed}}: legt die Gesamtzahl signifikanter Stellen fest. \newline \textbf{Mit \texttt{fixed}}: legt nur die Anzahl der Nachkommastellen fest. & \textbf{Dauerhaft (für den Stream).} \\ 
        \hline
        \texttt{fixed}           & Erzwingt Festkommadarstellung und deaktiviert die wissenschaftliche Notation. Reihenfolge zu \texttt{setprecision} ist egal. & \textbf{Dauerhaft.} \\ 
        \hline
\end{tabular}


\subsection{Namespaces}

Namespaces helfen, Namenskonflikte zu verhindern. 
Sie fügen ein Namensvorsatz zu allen Variablen. 
Können sehr mächtig sein, führt aber zu Komplexität

\cor{Besser mit kleine Buchstaben starten}

\lstinputlisting{code/namespace.cpp}