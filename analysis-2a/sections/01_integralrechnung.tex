\section{Integralrechnung}{493}

\subsection{Rechenregeln mit Integralen}{508-510}

\renewcommand{\arraystretch}{2.3}
\begin{tabular}{ll}
    Zerlegung:          & $\int \limits_{a}^{b} f_1(x) + f_2(x) \, \diff x 
                        = \int \limits_{a}^{b} f_1(x) \, \diff x + \int \limits_{a}^{b} f_2(x) \, \diff x$                                      \\
                        & $\int \limits_{a}^{c} f(x) \, \diff x = \int \limits_{a}^{b} f(x) \, \diff x + \int \limits_{b}^{c} f(x) \, \diff x$  \\
    Grenzen tauschen:   & $ \int \limits_{a}^{b} f(x) \, \diff x = -  \int \limits_{b}^{a} f(x) \, \diff x $                                    \\
    Gleiche Grenzen:    & $\int \limits_{a}^{a} f(x) \, \diff x = 0$
\end{tabular}		
\renewcommand{\arraystretch}{1}


\subsection{Integrationsregeln}{494 - 496}

\vspace{-0.2cm}

\begin{minipage}[t]{0.4\columnwidth}
    \subsubsection{Linearität}

    \vspace{-0.5cm}

    $$ \int \limits_{a}^{b} \alpha f(x) \, \diff x = \alpha \int \limits_{a}^{b} f(x) \, \diff x $$
\end{minipage}
\hfill
\begin{minipage}[t]{0.58\columnwidth}
    \subsubsection{Elementartransformation}

    $$ \int f(\alpha \, x + \beta) \, \diff x = \frac{1}{\alpha} \, F(\alpha \, x + \beta) + C$$
\end{minipage}


\subsubsection{Partielle Integration (Produktregel)}

\vspace{-0.5cm}

$$ \int f' \cdot g \, \diff x =  f \cdot g - \int f \cdot g' \, \diff x \qquad \qquad
\int f \cdot g' \, \diff x = f \cdot g - \int f' \cdot g \, \diff x $$

\textrightarrow\ Partielle Integration darf mehrfach angewendet werden. \\
\textbf{PI sollte wenn möglich vermieden werden! Produkte in Summen umschreiben!}


\subsubsection{Substitution}

\vspace{-0.2cm}

$$ \int f(x) \, dx = \int f(g(t)) \cdot g'(t) \, \diff t \qquad \text{ \textrightarrow\ siehe Beispiel} $$

\textbf{\crd{Integrationsgrenzen anpassen!} (ev. mit Umkehrfunktion)} 


\example{Substitution}

\begin{tabular}{ll}
    $\int \limits_0^{\sqrt{\pi}} x^3 \cdot \cos(x^2) \, \diff x$    & \cbl{Substitution: $a = x^2$} \crd{ \textbf{auch auf Grenzen anwenden!}}  \\
                                                                    & \cbl{$\diff a = 2 \, x \cdot \, \diff x$ \textrightarrow\ 
                                                                     $\diff x = \frac{\diff a}{2x} = \frac{\diff a}{2 \sqrt{a}} $}  \\
\end{tabular}

$\int \limits_{\crd{0}}^{\crd{\pi}} a \, \sqrt{a} \cdot \cos(a) \frac{\diff a}{2 \sqrt{a}} 
= \int \limits_{\crd{0}}^{\crd{\pi}} a \, \sqrt{a} \cdot \cos(a) \, \frac{1}{2} \frac{1}{\sqrt{a}} \, \diff a 
= \frac{1}{2} \int \limits_{\crd{0}}^{\crd{\pi}} a \cdot \cos(a) \, \diff a = ... $



\subsubsection{Universalsubstitution (Weierstrass) / Rationalisierung}

Rationale Terme aus $\sin(x)$ und $\cos(x)$ substituieren: (verknüpft durch +  -  *  : ) 

\smallskip

\begin{minipage}[c]{0.55\columnwidth}
    % \renewcommand{\arraystretch}{2}
    % \begin{tabular}{| c | c | c | c |}
    %     \hline
    %     $\sin(x)$               & $\cos(x)$                 & $\diff x$                     &   $t = \tan \Big( \frac{x}{2} \Big)$  \\
    %     \hline
    %     $\frac{2 t}{1 + t^2}$   & $\frac{1 - t^2}{1 + t^2}$ & $\frac{2}{1 + t^2} \diff t$   &                                       \\
    %     \hline
    % \end{tabular}
    % \renewcommand{\arraystretch}{1}

    \begin{tabular}{ c cc cc}
            \toprule
            $\sin(x)$               & & $\cos(x)$                 & & $\diff x$                     \\
            \midrule
            $\frac{2 t}{1 + t^2}$   & & $\frac{1 - t^2}{1 + t^2}$ & & $\frac{2}{1 + t^2} \diff t$   \\
            \bottomrule
    \end{tabular}

    \smallskip

    \textrightarrow\ $t = \tan \Big( \frac{x}{2} \Big)$
\end{minipage}
\hfill
\begin{minipage}[c]{0.42\columnwidth}
    \crd{Definitionsbereich: \\
         $-\pi < x < \pi$, da sonst $t \notin \mathbb{R}$}
\end{minipage}


\subsubsection{Spezielle Regeln (Faktor in Integral = Ableitung)}		

\vspace{-0.2cm}

\renewcommand{\arraystretch}{2.3}
\begin{tabular}{ll}
    Allg. Potenzregel   & $\int f'(x) \cdot f(x)^{\alpha} \, \diff x = \frac{f(x)^{\alpha + 1}}{\alpha + 1} + C \quad   (\alpha \neq -1)$   \\
    Allg. Log-Regel     & $\int f'(x) \cdot \frac{1}{f(x)} \, \diff x = \int \frac{f'(x)}{f(x)} \, \diff x = \ln(\vert f(x) \vert ) + C$    \\
    Allg. Exp-Regel     & $\int f'(x) \cdot \e^{f(x)} \, \diff x = \e^{f(x)} + C$
\end{tabular}				
\renewcommand{\arraystretch}{1}


\subsection{Uneigentliche Integrale}{520}

\subsubsection{Integrationsbereich auf x-Achse unbeschränkt ($\mathbb{D}_f$)}

\vspace{-0.2cm}

\renewcommand{\arraystretch}{2.3}
\begin{tabular}{lll}
    $\int \limits_a^{\infty} f(x) \, \diff x :=$            & $\lim \limits_{t \rightarrow \infty} \int \limits_a^t f(x) \, \diff x$                    & $a \in \mathbb{R}$ \quad $x \in [a; \infty)$      \\
    $\int \limits_{- \infty}^{b} f(x) \, \diff x :=$        & $\lim \limits_{t \rightarrow \infty} \int \limits_t^b f(x) \, \diff x$                    & $b \in \mathbb{R}$ \quad $x \in (- \infty; b]$    \\
    $\int \limits_{- \infty}^{\infty} f(x) \, \diff x :=$   & $\int \limits_{- \infty}^{C} f(x) \, \diff x + \int \limits_{C}^{\infty} f(x) \, \diff x$ & (Zerlegung)
\end{tabular}
\renewcommand{\arraystretch}{1}

\textbf{Wenn $\bm{f(x)}$ (stetig) nicht $\bm{\rightarrow}$ 0, dann liegt Divergenz vor!}


\subsubsection{Integrationsbereich auf y-Achse unbeschränkt ($\mathbb{W}_f)$}

\vspace{-0.2cm}

\renewcommand{\arraystretch}{2.7}
\begin{tabular}{ll}
    links ($x_0 = a$)           & $\int \limits_{a^+}^{b} f(x) \, \diff x = \lim \limits_{t \to a^+} \int \limits_t^b f(x) \, \diff x$                              \\
    rechts ($x_0 = b$)          & $\int \limits_{a}^{b^-} f(x) \diff x = \lim \limits_{t \to b^-} \int \limits_a^t f(x) \, \diff x$                                 \\
    zwischen ($x_0 \in (a;b))$  & $\int \limits_{a}^{x_0^-} f(x) \, \diff x = \int \limits_{a}^{x_0^-} f(x) \, \diff x + \int \limits_{x_0^+}^{b} f(x) \, \diff x$  \\
    beide ($x_0 = a; x_0 = b$)  & $\int \limits_{a^+}^{b^-} f(x) \, \diff x = \int \limits_{a^+}^{C} f(x) \, \diff x + \int \limits_{C}^{b^-} f(x) \, \diff x$ 
\end{tabular}
\renewcommand{\arraystretch}{1}

\smallskip

\textbf{Generell gilt: Bei Unstetigkeitsstellen wird Integral zerlegt!}


\subsection{Referenz uneigentlicher Integrale}

\vspace{-0.2cm}

\begin{minipage}[c]{0.48\columnwidth}
    $$ \int \limits_{a}^{\infty} \frac{1}{x^{\alpha}} \, \diff x = 
    \begin{cases}			
        \infty      & \crd{\text{Div}} \quad \alpha \in (0; 1]         \\
        \mathbb{R}  & \cbl{\text{Konv}} \quad \alpha > 1
    \end{cases}$$  
\end{minipage}
\hfill
\begin{minipage}[c]{0.48\columnwidth}
    $$ \int \limits_{0^+}^{b > 0} \frac{1}{x^{\beta}} \, \diff x = 
    \begin{cases}		
        \mathbb{R}  & \cbl{\text{Konv}} \quad 0 \leq \beta < 1  \\
        \infty      & \crd{\text{Div}} \qquad \beta > 1         \\
        \infty      & \crd{\text{Div}} \qquad \beta = 1
    \end{cases} $$ 
\end{minipage}


\subsection{Majoranten- / Minorantenprinzip}

\subsubsection{Majorantenprinzip ("Grösser-Gleich-Term")}

$\crd{\vert} f(x) \crd{\vert} \leq g(x)$ \qquad $g(x)$ = Majorante \qquad $D_f = D_g$ zw. Grenzen von Integral

$$ \text{Wenn } \int \limits_{...}^{...} g(x) \, \diff x \text{ \cbl{konv.} \textrightarrow\ } 
\int \limits_{...}^{...} \vert f(x) \vert \, \diff x \text{ \cbl{ konv.} \textrightarrow\ } 
\int \limits_{...}^{...} f(x) \, \diff x  \text{ \cbl{ konv.}} $$


\subsubsection{Minorantenprinzip ("Kleiner-Gleich-Term")}

$0 \leq f(x)$ \quad $0 \leq g(x) \leq f(x)$ \quad $g(x)$ = Minorante \quad $D_f = D_g$ zw. Grenzen von Integral

$$ \text{Wenn } \int \limits_{...}^{...} g(x) \, \diff x \text{ \crd{div.} \textrightarrow\ } 
\int \limits_{...}^{...} f(x) \, \diff x \text{ \crd{ div.}} $$


\subsection{Integraltabelle}{495}
        
\begin{center}
    \renewcommand{\arraystretch}{1.6}
    \begin{tabular}{| c  c  c |}
        \hline
        \textbf{Ableitung} $\bm{\frac{\diff f}{\diff x}}$   & \textbf{Funktion} $\bm{f(x)}$ & \textbf{Stammfunktion} $\bm{F(x)}$    \\ 
        \hline
        $0$                                                 & $c \; (c \in \mathbb{R})$     & $c \cdot x$                           \\ 
        $c$                                                 & $c \cdot x$                   & $c \frac{x^2}{2}$                     \\
        $a \cdot x^{a-1} $                                  & $x^a \; (a \in \mathbb{R}\diagdown \lbrace -1 \rbrace)$ &  $\frac{x^{a+1}}{a+1} $ \\
        $- \frac{1}{x^2} $                                  & $\frac{1}{x}$                 & $\ln(\vert x \vert)$                  \\ 
        \hline
        $\e^x$                                              & $\e^x$                        & $\e^x$                                \\ 
        $a \, \e^{ax}$                                      & $\e^{ax}$                     & $\frac{1}{a} \e^{ax}$                 \\ 
        $a^x \, \ln(\vert a \vert)$                         & $a^x$                         & $\frac{a^x}{\ln(\vert a \vert)}$      \\
        \hline
        $\frac{1}{x}$                                       & $\ln(\vert x \vert)$ &        $x \, ( \ln(\vert x \vert) - 1) $       \\
        \hline
        $\cos(x)$                                           & $\sin(x)$                     & $- \cos(x)$                           \\
        $- \sin(x)$                                         & $\cos(x)$                     & $\sin(x)$                             \\
        $1 + \tan^2(x)$                                     & $\tan(x)$                     & $- \ln(\vert \cos(x) \vert)$          \\
        \hline
        $\cosh(x)$                                          & $\sinh(x)$                    & $\cosh(x)$                            \\
        $\sinh(x)$                                          & $\cosh(x)$                    & $\sinh(x)$                            \\
        $1 - \tanh^2(x)$                                    & $\tanh(x)$                    & $\ln(\cosh(x))$                       \\
        \hline
        $\frac{1}{\sqrt{1 - x^2}}$                          & $\arcsin(x)$                  & $x \cdot \arcsin(x) + \sqrt{1- x^2 }$ \\
        $- \frac{1}{\sqrt{1 - x^2}}$                        & $\arccos(x)$                  & $x \cdot \arccos(x) - \sqrt{1- x^2 }$ \\
        $\frac{1}{1 + x^2}$                                 & $\arctan(x)$                  & $x \cdot \arctan(x) - \frac{1}{2} \ln(1 + x^2) $ \\
        \hline
        $\frac{1}{\sqrt{x^2 + 1}}$                          & $\mathrm{arsinh}(x)$          & $x \cdot \mathrm{arsinh}(x) - \sqrt{x^2 +1 } $ \\
        $\frac{1}{\sqrt{x^2 - 1}}$                          & $\mathrm{arcosh}(x)$          & $x \cdot \mathrm{arcosh}(x) - \sqrt{x^2 - 1} $ \\	
        $\frac{1}{1- x^2}$                                  & $\mathrm{artanh}(x)$          & $x \cdot \mathrm{artanh}(x) - \frac{1}{2} \ln(1-x^2) $ \\
        \hline
    \end{tabular}
    \renewcommand{\arraystretch}{1}
\end{center}