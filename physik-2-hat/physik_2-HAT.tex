% ========================================= TEMPLATE INFO ========================================
%
% Author:       P4ntomime
% Version:      1.0.0
% Last updated: 2024-02-18
% Brief:        A LaTeX template for summaries. See README.md for more information.
% 
% ================================================================================================
\documentclass[8pt, a4paper, twoside]{extarticle}
% Font size:    8pt
% Paper size:   A4
% style:        twoside (needed, so odd and even pages have different margins)
% orientation:  portrait. (use 'landscape' for landscape orientation)


% ========================================= DOCUMENT INFO =========================================
\def\title{Physik 2 - HAT}           				% title
\def\shorttitle{Ph2 - HAT} 							% short title (displayed as PDF title)
\def\dozent{Dr. David Sourlier}         			% lecturer
\def\semester{FS 2023}    	 						% semester
\def\author{Simone Stitz}        					% author(s)
\def\repo{https://gitlab.com/sstitz/physik-2-hat}   % repository link
\def\version{2.0.\today}    						% version
\def\pagelimit{30}           						% page limit -> causes pages after limit to be red
\def\titleoption{compact}   						% options: ultra compact, compact, normal
\def\enableToC{true}

% ================================= PACKAGES, SETUP AND COMMANDS ==================================
\input{preamble.tex}


% =========================================== DOCUMENT ============================================
\begin{document}
    \begin{layout}
		\part{Hydrostatik / Hydrodynamik}
        \section{Hydrostatik}

\subsection{Festkörper, Flüssigkeit, Gas}

\subsubsection{Festkörper}

\begin{itemize}
    \item Kein Fluid
    \item Festes Volumen; feste Gestalt
    \item Moleküle / Atome befinden sich in regelmässiger Gitter-Anordnung
    \item Inkompressibel (sehr schlecht komprimierbar)
    \item Kraft: Weiterleitung (längs ihrer Wirkungslinie)
    \item Druck: Verstärkung 
\end{itemize}


\subsubsection{Ideale Flüssigkeit}

\begin{itemize}
    \item Fluid
    \item Festes Volumen; keine feste Gestalt
    \item Moleküle / Atome bewegen sich chaotisch aneinander vorbei
    \item Moleküle / Atome füllen den Raum aus / berühren sich
    \item Inkompressibel (schlecht komprimierbar)
    \item Reibungsfrei (keine Scherkräfte)
    \item Kraft: Verstärkung
    \item Druck: Weiterleitung (gleichmässig)
\end{itemize}


\subsubsection{Gas}

\begin{itemize}
    \item Fluid
    \item Kein festes Volumen; keine feste Gestalt
    \item Moleküle / Atome fliegen mit hoher Geschwindigkeit durch den Raum
    \item Es gibt sehr viel Zwischenraum 
    \item Moleküle / Atome führen bei Zusammenstoss unter sich oder mit Gefässwand elestische Stösse aus
    \item Kompressibel (gut komprimierbar)
    \item Reibungsfrei (keine Scherkräfte)
\end{itemize}


\subsection[Druck / Schubspannung]{Druck $\bm{p}$ / Schubspannung $\bm{\tau}$}

\textbf{Druck ist eine skalare Grösse (hat keine Richtung)} 

$$\boxed{ p = \frac{F_{\perp}}{A} } \qquad \qquad \boxed{ \tau = \frac{F_{\parallel}}{A} } $$

\begin{tabular}{c l c}
    $p$             & Druck                         & $[p] = \pascal = \frac{\newton}{\meter^2}$    \\
    $\tau$          & Schubspannung (Scherkraft)    & $[\tau] = \newton$                            \\
    $F_{\perp}$     & Kraft senkrecht zu $A$        & $[F_{\perp}] = \newton$                       \\
    $F_{\parallel}$ & Kraft parallel zu $A$         & $[F_{\parallel}] = \newton$                   \\
    $A$             & Fläche                        & $[A] = \meter^2$
\end{tabular}

\medskip

\textbf{In abgeschlossenen, miteinander verbundenen Systemen herrscht ein Druck-Gleichgewicht!} 

$$ \boxed{ p_1 = p_2  \quad \Leftrightarrow  \quad \frac{F_1}{A_1} = \frac{F_2}{A_2} }$$



\subsubsection{Weitere Einheiten von Druck}

\vspace{-0.2cm}
$$ \bm{1 \, \bbar = 10^5 \pascal} \qquad \text{Absolutdruck: Vergleich zu Vakuum} $$

\begin{tabular}{ll}
    $1 \, \hecto \pascal$   & $= 100 \, \pascal = 1 \, \milli \bbar $                                       \\
    $1 \, \mathrm{at}$      & $= 1 \, \mathrm{kp \cdot \centi \meter^{-2}} = 9.81 \cdot 10^4 \, \pascal$    \\
    $1 \, \mathrm{at"u}$    & $= 1 \, \mathrm{at}$ (Überdruck; Vergleich zu normalem Luftdruck)             \\
    $1 \, \mathrm{Torr}$    & $= \frac{1}{760} \, \mathrm{at}$ (1 mm-Hg-Säule)
\end{tabular}


\subsection{Kompression}

\vspace{-0.2cm}

$$ \boxed{ \text{Flüssigkeiten:} \qquad \Delta p = \frac{1}{\kappa} \cdot - \frac{\Delta V}{V} = K \cdot - \frac{\Delta V}{V} } $$  
$$ \boxed{ \text{Gase:} \qquad \Delta p = p(h) - p_0 = \frac{1}{\kappa_T} \cdot - \frac{\Delta V}{V} } $$


\begin{tabular}{c l c}
    $\Delta p$              & Druckerhöhung                     & $[\Delta p] = \pascal = \frac{\newton}{\meter^2}$     \\
    $\kappa$                & Kompressibilität (Flüssigkeit)    & $[\kappa] = \frac{1}{\pascal}$                        \\
    $K = \frac{1}{\kappa}$  & Kompressionsmodul                 & $[K] = \pascal$                                       \\
    $\kappa_T$              & Kompressibilität (Gas)            & $[\kappa_T] = \frac{1}{\pascal}$                      \\
    $- \frac{\Delta V}{V}$  & realtive Volumen-Abnahme          & $\Big[ \frac{\Delta V}{V} \Big] = 1$ 
\end{tabular}


\subsection[Dichte]{Dichte $\bm{\rho}$}


\begin{minipage}[c]{0.48\columnwidth}
    $$ \boxed{ \rho = \frac{m}{V} \quad \Leftrightarrow \quad m = \rho \cdot V } $$	 
\end{minipage}
\hfill
\begin{minipage}[c]{0.48\columnwidth}
    \begin{tabular}{c l c}
        $\rho$  & Dichte    & $[\rho] = \frac{\kilogram}{\meter^3}$ \\
        $m$     & Masse     & $[m] = \kilogram$                     \\
        $V$     & Volumen   & $[V] = \meter^3$
    \end{tabular}
\end{minipage}


\subsubsection{Wichtige Dichten}	

\begin{tabular}{l c l}
    $\rho_{\rm Wasser} = 1000 \, \frac{\kilogram}{\meter^3}$    & &  $\rho_{\rm Luft} = 1.2 \, \frac{\kilogram}{\meter^3}$
   
\end{tabular}


\subsection{Boyle-Mariotte}	
\textbf{Das Gesetz von Boyle-Mariotte beschreibt die Kompressibilität von Gasen.} \\
\textbf{ \textrightarrow\ Das Gesetz gilt nur bei konstanter Temperatur!} 

$$ \boxed{ p_1 \cdot V_1 = p_2 \cdot V_2 = \, \const \quad \Leftrightarrow \quad \frac{p_1}{p_2} = \frac{\rho_1}{\rho_2} } $$ 

\renewcommand{\arraystretch}{1.3}
\begin{tabular}{c l c}
    $\rho_x$    & Gas-Dichte    & $[\rho_x] = \frac{\kilogram}{\meter^3}$   \\
    $p_x$       & Gas-Druck     & $[p_x] = \pascal $                        \\
    $V_x$       & Volumen       & $[V_x] = \meter^3$
\end{tabular}
\renewcommand{\arraystretch}{1}


\subsection{Hydrostatischer Druck (Schweredruck)}

\textbf{Gilt nur für Flüssigkeiten!} 

$$ \boxed{ p = \rho \cdot g \cdot h }$$	

\renewcommand{\arraystretch}{1.3}
\begin{tabular}{c l c}
    $\rho$  & Dichte der Flüssigkeit                                    & $[\rho] = \frac{\kilogram}{\meter^3}$ \\
    $h$     & Höhe \textbf{unter} der Flüssigkeits-Oberfläche           & $[h] = \meter$                        \\
    $g$     & Erdbeschleunigung $g = 9.81 \, \frac{\meter}{\second^2}$  & $[g] = \frac{\meter}{\second^2}$
\end{tabular}
\renewcommand{\arraystretch}{1}

\smallskip

\textbf{Der Druck ist nur von der Höhe der darüberliegenden Flüssigkeit abhängig, nicht von deren Volumen oder Gewicht.}


\subsection{Barometrische Höhenformel (Gase)}

\vspace{-0.2cm}

$$ \boxed{ p(h) = p_0 \cdot e^ {- \frac{\rho_0}{p_0} \cdot g \cdot h} }$$	


\begin{tabular}{c l c}
    $p(h)$      & Schweredruck des Gases bei Höhe $h$                                       & $[p(h)] = \pascal$                        \\
    $p_0$       & Luftdruck auf Meereshöhe $p_0 = 10^5 \, \pascal$                          & $[p_0] = \pascal$                         \\ 
    $\rho_0$    & Luft-Dichte auf Meereshöhe $\rho_0 = 1.2 \, \frac{\kilogram}{\meter^3}$   & $[\rho_0] = \frac{\kilogram}{\meter^3}$   \\
    $h$         & Höhe über Meer                                                            & $[h] = \meter$                            \\
    $g$         & Erdbeschleunigung $g = 9.81 \, \frac{\meter}{\second^2}$                  & $[g] = \frac{\meter}{\second^2}$    
\end{tabular}


\subsection{Statischer Auftrieb (Fluid)}

Der Auftrieb eines Körpers entspricht dem Gewicht der von ihm verdrängten Flüssigkeit (Archimedes). 


\begin{minipage}{0.6\columnwidth}
    $$ \boxed{ F_A = \rho_{\rm Fl} \cdot V_K \cdot g } $$
    $$ \boxed{ F_A = F_{\rm G,Fl} = m_{\rm Fl} \cdot g = \rho_{\rm Fl} \cdot V_K \cdot g } $$
\end{minipage}
\hfill
\begin{minipage}{0.28\columnwidth}
    \includegraphics[width=\columnwidth]{images/auftrieb.jpg}
\end{minipage}

\smallskip

\renewcommand{\arraystretch}{1.3}
\begin{tabular}{c l c}
    $F_A$           & Auftriebskraft                                        & $[F_A] = \newton$                                 \\
    $\rho_{\rm Fl}$ & Dichte \textbf{verdrängtes Fluid}                     & $[\rho_{\rm Fl}] = \frac{\kilogram}{\meter^3}$    \\
    $V_K$           & Verdrängtes Fluid-Volumen                             & $[V_K] = \meter^3$                                \\
    $g$         & Erdbeschleunigung $g = 9.81 \, \frac{\meter}{\second^2}$  & $[g] = \frac{\meter}{\second^2}$                  \\
    $m_{\rm Fl}$    & Masse des \textbf{verdrängten Fluids}                 & $[m_{\rm Fl}] = \kilogram$                        \\
    $F_{\rm G,Fl}$  & Gewichtskraft \textbf{verdrängtes Fluid}              & $[F_{\rm G,Fl}] = \newton$
\end{tabular}
\renewcommand{\arraystretch}{1}


\subsection[Oberflächenspannung]{Oberflächenspannung $\bm{\sigma}$}

\begin{minipage}[c]{0.3\columnwidth}
    $$ \boxed{ \sigma := \frac{F}{l} } $$ 
\end{minipage}
\hfill
\begin{minipage}[c]{0.65\columnwidth}
    \renewcommand{\arraystretch}{1.3}
    \begin{tabular}{c l c}
    $\sigma$    & Oberflächenspannung   & $[\sigma] = \frac{\newton}{\meter}$   \\
    $F$         & Kraft                 & $[F] = \newton$                       \\
    $l$         & Länge                 & $[l] = \meter$
    \end{tabular}
    \renewcommand{\arraystretch}{1}
\end{minipage}

\medskip

\textbf{Die Länge $\bm{l}$ entspricht der gesamten Berührungslänge zwischen Flüssigkeit und Festkorper / Gas}

\smallskip

\begin{tabular}{ll c|c ll}
    Zylinder & $l = 2 \, \pi \, r$ & &
    Lamellen & $l = 2 \, b$  (beidseitig!)
\end{tabular}


\subsection[Kapillarität]{Kapillarität $\bm{h}$}

\vspace{-0.2cm}

$$\boxed{  h = \frac{2 \cdot \sigma}{\rho \cdot g \cdot r} = \frac{\sigma}{\rho \cdot g \cdot d} }$$ 


\begin{tabular}{c l c}
    $\sigma$    & Totale Grenzflächenspannung   & $[\sigma] = \frac{\newton}{\meter}$   \\
    $\rho$      & Dichte der Flüssigkeit        & $[\rho] = \frac{\kilogram}{\meter^3}$ \\
    $r$         & Radius der Kapillare          & $[r] = \meter$                        \\
    $d$         & Durchmesser der Kapillare     & $[r] = \meter$ 
\end{tabular}

\medskip

\begin{minipage}[b]{0.48\columnwidth}
    \begin{center}
        \includegraphics[width=0.3\columnwidth]{images/kapillaritaet_benetzend.png}

        benetzend
    \end{center}
\end{minipage}
\hfill
\begin{minipage}[b]{0.48\columnwidth}
    \begin{center}
        \includegraphics[width=0.3\columnwidth]{images/kapillaritaet_nicht_benetzend.png}

        nicht benetzend
    \end{center}
\end{minipage}


\columnbreak


\subsection[Druck in Seifenblase]{Druck in Seifenblase $\bm{p}$}

\begin{minipage}[c]{0.3\columnwidth}
    $$ \boxed{ p = \frac{2 \cdot \sigma}{r} } $$ 
\end{minipage}
\hfill
\begin{minipage}[c]{0.66\columnwidth}
    \renewcommand{\arraystretch}{1.3}
    \begin{tabular}{c l c}
        $\sigma$    & Oberflächenspannung       & $[\sigma] = \frac{\newton}{\meter}$ \\
        $r$         & Radius der Seifenblase    & $[r] = \meter$
    \end{tabular}
    \renewcommand{\arraystretch}{1}
\end{minipage}

\smallskip


		\section{Hydrodynamik - Ideale Fluide}

\textbf{Ideale Fluide nehmen keine Scherkräfte auf (keine Reibung) und sind inkompressibel.}


\subsection{Stromlinien-Modell}

\begin{itemize}
	\item Stromlinien zeigen Geschwindigkeit des Fluids
	\item \textbf{Dichte} Stromlinien bedeutet \textbf{hohe} Geschwindigkeit
	\item \textbf{Dünne} Stromlinien bedeutet \textbf{niedrige} Geschwindigkeit 
	\item Stationär: Stromlinien schneiden sich nicht 
\end{itemize}


\subsection{Kontinuitätsgleichung}

\begin{center}
	\includegraphics[width=0.7\columnwidth]{images/Kontinuitaet.png}
\end{center}

\vspace{-0.5cm}

$$ \boxed{ \frac{\Delta V}{\Delta t} = \dot{V} = A \cdot v = \const  \quad \Leftrightarrow \quad  A_1 \cdot v_1 = A_2 \cdot v_2 = \frac{\Delta V}{\Delta t} = \dot{V}} $$ 

\renewcommand{\arraystretch}{1.3}
\begin{tabular}{c l c}
	$\Delta V$	& Volumenänderung 					& $[\Delta V] = \meter^3$					\\
	$\Delta t$ 	& Zeitänderung 						& $[\Delta t] = \second$  					\\
	$\dot{V}$ 	& Volumenstrom (Volumen pro Zeit) 	& $[\dot{V}] = \frac{\meter^3}{\second}$	\\
	$A_x$ 		& Querschnittsfläche 				& $[A_x] = \meter^2$ 						\\
	$v_x$ 		& Geschwindigkeit der Flüssigkeit 	& $[v_x] = \frac{\meter}{\second}$
\end{tabular}
\renewcommand{\arraystretch}{1}

\medskip

\textrightarrow\ Gilt auch für Gase, wenn $v \ll v_{\rm Schall}$


\subsection{Bernoulli-Gleichung}

\begin{minipage}[c]{0.38\columnwidth}
	\raggedright
	Die Bernoulli-Gleichung beschreibt ein \myul{bewegtes} Fluid

	$$ \underbrace{ p + \rho \cdot g \cdot h }_{\substack{\mathrm{statisch}}} 
	+ \underbrace{ \frac{1}{2} \, \rho \cdot v^2 }_{\substack{\mathrm{dynamisch}}} = \const $$
\end{minipage}
\hfill
\begin{minipage}[c]{0.6\columnwidth}
	\includegraphics[width=\columnwidth]{images/Bernoulli.png}
\end{minipage}

\smallskip

$$ \boxed{ p_1 +  \rho \cdot g \cdot h_1 + \frac{1}{2} \, \rho \cdot v_1^2 = p_2 +  \rho \cdot g \cdot h_2 + \frac{1}{2} \, \rho \cdot v_2^2 } $$

\medskip


\begin{minipage}[t]{0.48\columnwidth}
	\subsubsection{Spezialfall: Horizontal}

	$$ \boxed{ p + \frac{1}{2} \, \rho \cdot v^2 = \const } $$
\end{minipage}
\hfill
\begin{minipage}[t]{0.48\columnwidth}
	\subsubsection{Spezialfall: Statik}

	$$ \boxed{ p + \rho \, \cdot g \cdot h =  \const} $$
\end{minipage}

\medskip


\subsubsection{Hydrodynamisches Paradoxon}

\textbf{Je grösser die Strömungsgeschwindigkeit, desto kleiner der Druck} \\
\textrightarrow\ Gegen jede Intuition!




\subsection{Bernoulli-Gleichung und Energieerhaltung}

Die in der Bernoulli-Gleichung vorkommenden Terme können als \myul{Energie pro Volumen} betrachtet werden.

\vspace{-0.3cm}

\begin{align*}
	E_{\rm Mech}	&= \text{Elastische Energie } + \text{ pot. Energie } + \text{ kin. Energie} \\
					&= p \cdot V + m \cdot g \cdot h + \frac{1}{2} \, m \cdot v^2 = \const
\end{align*}



Wenn durch das Volumen dividiert wird erhält man: 

\vspace{-0.3cm}

\begin{align*}
	\frac{E_{\rm Mech}}{\text{Volumen}}	&= \frac{\text{elatische Energie}}{\text{Volumen}} + \frac{\text{pot. Energie}}{\text{Volumen}} + \frac{\text{kin. Energie}}{\text{Volumen}} \\
										&= p + \rho \cdot g \cdot h + \frac{1}{2} \, \rho \cdot v^2 = \const
\end{align*}


Bei einer horizontalen Strömung entfällt die pot. Energie (pro Volumen)

\vspace{-0.3cm}

\begin{align*}
	\frac{E_{\rm Mech}}{\text{Volumen}}	&= \frac{\text{elatische Energie}}{\text{Volumen}} + \frac{\text{kin. Energie}}{\text{Volumen}} \\
										&= p + \frac{1}{2} \, \rho \cdot v^2 = \const
\end{align*}


\columnbreak


		\section{Hydrodynamik - Reale Fluide}

\textbf{Reale Fluide nehmen Scherkräfte auf (Reibung)}

\subsection{Newton'sches Reibungs-Gesetz}

\vspace{-0.2cm}

$$ \boxed{ \tau = \eta \cdot \frac{v}{d} }  \qquad  \boxed{ \tau = \eta \cdot \frac{\diff v}{\diff z} } $$

\renewcommand{\arraystretch}{1.3}
\begin{tabular}{c l c}
    $\tau$                      & Schubspannung                             & $[\tau] = \newton$                                \\
    $\eta$                      & Dynmaische Zähigkeit (Viskosität)         & $[\eta] = \pascal \, \second$                     \\
    $v$                         & Geschwindigkeitsdifferenz zw. Auflagen    & $[v] = \frac{\meter}{\second}$                    \\
    $z$                         & Richtung senkrecht zur Verschiebung       & $[z] = \meter$                                    \\
    $d$                         & Distand zwischen den Auflagen             & $[d] = \meter$                                    \\
    $\frac{\diff v}{\diff z}$   & Geschwindigkeits-Gradient in z-Richtung   & $[\frac{\diff v}{\diff z}] = \frac{1}{\second}$ 
\end{tabular}
\renewcommand{\arraystretch}{1}

\medskip


\para{Werte für $\bm{\eta}$}

\renewcommand{\arraystretch}{1.5}
\begin{tabular}{ll c|c ll}
    $\eta_{\rm Luft}$   & $:= 2 \cdot 10^{-5} \, \pascal \, \second$    & &  $\eta_{\rm Oel}$   & $:= 0.1 \, \pascal \, \second$ \; bis \; $1 \, \pascal \, \second$ \\
    $\eta_{\rm Wasser}$ & $:= 10^{-2} \, \pascal \, \second$            & & \\
\end{tabular}
\renewcommand{\arraystretch}{1}

\medskip


\subsubsection{Kinematische Zähigkeit $\bm{\nu}$}

\begin{minipage}[c]{0.3\columnwidth}
    $$ \boxed{ \nu = \frac{\eta}{\rho} } $$
\end{minipage}
\hfill
\begin{minipage}[c]{0.68\columnwidth}
    \renewcommand{\arraystretch}{1.5}
    \begin{tabular}{c l c}
        $\nu$   & Kinemtaische Zähigkeit    & $[\nu] = \frac{\meter^2}{\second}$    \\
        $\rho$  & Dichte                    & $[\rho] = \frac{\kilogram}{\meter^3}$ \\	
    \end{tabular}
    \renewcommand{\arraystretch}{1.5}
\end{minipage}


\subsection[Stokes'sche Reibung]{Stokes'sche Reibung $\bm{F_R}$}

Verwendet für z.B. Kugel in Öl oder fallende Wassertropfen

$$ \boxed{ F_R = 6 \cdot \pi \cdot \eta \cdot R \cdot v } $$

\begin{tabular}{c l c}
    $F_R$   & Reibungskraft                     & $[F_R] = \newton$                         \\
    $\eta$  & Dynamische Zähigkeit (Viskosität) & $[\eta] = \mathrm{\pascal \, \second}$    \\
    $R$     & Kugelradius                       & $[R] = \meter$                            \\
    $v$     & Geschwindigkeit                   & $[v] = \frac{\meter}{\second}$		
\end{tabular}


\subsubsection{Kugelfall-Viskosimeter}

\begin{minipage}{0.35\columnwidth}
    \includegraphics[width=\columnwidth]{images/kugelfall-viskosimeter.png}
\end{minipage}
\hfill
\begin{minipage}{0.6\columnwidth}
    \raggedright

    Auf eine Kugel, welche in einer Flüssigkeit hinabgleitet wirken folgende Kräfte: 

    \medskip

    \begin{tabular}{ll}
        $F_G$   & Gewichtskraft         \\
        $F_A$   & statischer Auftrieb   \\
        $F_R$   & Stokes'sche Reibung
    \end{tabular}

    \medskip

    Ansatz zum Lösen von Aufgaben: \textbf{Kräftegleichgewicht}
\end{minipage}



\subsection{Hagen-Poiseuille}

Beschreibung von \myul{laminaren} Strömungen in einem \myul{runden Rohr} \textrightarrow\ Schichtströmung

\subsubsection{Gesetz von Hagen-Poiseuille}

\vspace{-0.2cm}

$$ \boxed{ \dot{V} = \frac{\pi \cdot \Delta \, p \cdot R^4}{8 \cdot \eta \cdot l} } $$


\subsubsection{Geschwindigkeitsverteilung von $\bm{r=0}$ bis $\bm{R}$}

\vspace{-0.2cm}

$$ \boxed{ v(r) = \frac{1}{4 \cdot \eta} \cdot \frac{\Delta \, p}{l} \, (R^2 - r^2) } $$

\renewcommand{\arraystretch}{1.3}
\begin{tabular}{c l c}
    $v(r)$                              & Fliessgeschwindigkeit beim Radius $r$ & $[v(r)] = \frac{\meter}{\second}$         \\
    $r$                                 & betrachteter Radius                   & $[r] = \meter$                            \\
    $\eta$                              & Dynamische Zähigkeit (Viskosität)     & $[\eta] = \pascal \, \second$             \\
    $R$                                 & Rohr-(Innen)Radius                    & $[R] = \meter$                            \\
    $\Delta \, p$                       & Druckdifferenz                        & $[\Delta \, p] = \pascal$                 \\
    $\dot{V} = \frac{\diff V}{\diff t}$ &  Volumenstrom                         & $[\dot{V}] = \frac{\meter^3}{\second}$	\\
    $l$                                 & Länge des Rohrs                       & $[l] = \meter$
\end{tabular}
\renewcommand{\arraystretch}{1}

\columnbreak


\subsection[Reynolds-Zahl]{Reynolds-Zahl $\bm{\Rey}$}

Die Reynoldszahl ist ein Richtmass für die Wirbelbildung.

\smallskip
\begin{itemize}
    \item Druck-Differenz (Bernoulli) begünstigt Wirbelbildung
    \item Innere Reibung (Schubspannung) verhindert Wirbelbildung 
\end{itemize}

$$ \boxed{ \Rey = \frac{\Delta \, p}{\tau} = \frac{\rho \cdot \overline{v} \cdot d}{\eta}} \qquad \quad \text{mit } \; \overline{v} = \frac{\dot{V}}{A} $$


\begin{tabular}{c l c}
    $\Rey$          & Reynolds-Zahl                         & $[\Rey] = 1$                              \\
    $\eta$          & Dynamische Zähigkeit (Viskosität)     & $[\eta] = \pascal \, \second$             \\
    $\overline{v}$  & Mittlere Geschwindigkeit              & $[\overline{v}] = \frac{\meter}{\second}$ \\
    $d$             & Typische Dimension (Rohrdurchmesser)  & $[d] = \meter$                            \\
    $\Delta \, p$   & Druckdifferenz                        & $[\Delta p] = \pascal$                    \\
    $\tau$          & Schubspannung                         & $[\tau] = \newton$
\end{tabular}

\medskip

\textbf{Sobald die Reynolds-Zahl $\bm{\Rey}$ grösser ist als ein kritischer Wert bilden sich Wirbel.} \\
\textrightarrow\ Rohr:  $\Rey_{\rm kritisch} \approx 2320$


\subsubsection{Ähnlichkeitsgesetz}

\begin{itemize}
    \item Reynolds-Zahl dient auch richtigem Vergleich von Modellversuchen \\
        \textrightarrow\ Gleiche Reynolds-Zahl bedeutet gleiches Verhalten
    \item  Gleiche Reynolds-Zahl bedeutet auch gleiche relative Grenzschicht-Dicke $D$ (siehe Abschnitt \ref{Prandl'sche Grenzschicht-Dicke})
\end{itemize}


\subsection{Turbulente / Laminare Rohrströmung}

\subsubsection{Hilfe, um Reynoldszahl zu bestimmen (laminar)}

\vspace{-0.2cm}

$$ \boxed{ \Delta p = 32 \cdot \eta \cdot l \cdot \frac{v}{d^2} }  $$


\subsubsection{Druckunterschied in laminare / turbulente Strömung}

$$ \lambda_{\rm turbulent} = \frac{0.316}{\sqrt[4]{\Rey}}  \qquad \qquad \lambda_{\rm laminar} = \frac{64}{\Rey} $$
$$ \boxed{ \Rightarrow \Delta p_x = \lambda_x \frac{l}{d} \cdot \frac{\rho}{2} \cdot v^2 } $$


\begin{tabular}{c l c}
    $\Delta \, p_x$ & Druckdifferenz (laminar / turbulent)  & $[\Delta p] = \pascal$                \\
    $\eta$          & Dynamische Zähigkeit (Viskosität)     & $[\eta] = \pascal \, \second$         \\
    $l$             & Rohr-Länge                            & $[l] = \meter$                        \\
    $v$             & Fliess-Geschwindigkeit                & $[v] = \frac{\meter}{\second}$        \\
    $d$             & Rohr-Durchmesser                      & $[d] = \meter$                        \\		
    $\rho$          & Dichte des Fluids                     & $[\rho] = \frac{\kilogram}{\meter^3}$ \\
    $\Rey$          & Reynolds-Zahl                         & $[\Rey] = 1$
\end{tabular}


\subsubsection{Unbekannt / Gemischt (Pratische Anwendung)}

Vorgehen, wenn man nicht weiss, ob sich Wirbel bilden oder nicht

\smallskip

\begin{enumerate}
    \item Laminar rechnen (um fehlenden Parameter $\rho, \; v, \; d, \text{ oder } \eta$ zu bestimmen) 
    \item Aus Resultat Reynolds-Zahl $\Rey$ berechnen
    \item Mit kritischer Reynolds-Zahl $\Rey_{\rm kritisch}$ vergleichen
    \item Beim \textbf{Überschreiten} \textrightarrow\ Turbulent rechnen!
\end{enumerate}


\subsection[Prandl'sche Grenzschicht-Dicke]{ $\bm{D}$}
\label{Prandl'sche Grenzschicht-Dicke}

\begin{minipage}[t]{0.4\columnwidth}
    \includegraphics[width=\columnwidth, align=t]{images/prandl.png}
\end{minipage}
\hfill
\begin{minipage}[t]{0.58\columnwidth}
    Prandl'sche Grenzschicht-Dicke $D$ beschreibt, in welcher \textbf{Distanz} die \textbf{Geschwindigkeit} eines laminar bewegten Teils
    (z.B. ein Flugzeugflügel) \textbf{Null} ist. 

    \medskip

    Die Geschwindigkeit innerhalb der Grendschicht $D$ nimmt von Teil bis hin zum äussersten Rand \textbf{linear} ab.

    $$\boxed{ D = \sqrt{\frac{\eta}{\rho} \cdot \frac{l}{v}} } $$
\end{minipage}

\medskip

\begin{tabular}{c l c}
    $D$     & Prandl'sche Grenzschicht-Dicke                    & $[D] = \meter$                        \\
    $\eta$  & Dynamische Zähigkeit (Viskosität)                 & $[\eta] = \pascal \, \second$         \\
    $\rho$  & Dichte des Fluids                                 & $[\rho] = \frac{\kilogram}{\meter^3}$ \\
    $l$     & Länge des bewegten Teils (in Richtung von $v$)    & $[l] = \meter$                        \\
    $v$     & Geschwindigkeit                                   & $[v] = \frac{\meter}{\second}$
\end{tabular}


\subsection{Bernoulli-Gleichung mit innerer Reibung}

\vspace{-0.2cm}

$$ \boxed{ p_1 +  \rho \cdot g \cdot h_1 + \frac{1}{2} \, \crd{\alpha_1} \cdot \rho \cdot v_1^2
= p_2 +  \rho \cdot g \cdot h_2 + \frac{1}{2} \, \crd{\alpha_2} \cdot \rho \cdot v_2^2 \crd{+ \Delta \, p_v} } $$

\renewcommand{\arraystretch}{1.6}
\begin{ctabular}{c| c |c}
                                    & laminar                                                   & turbulent                                                 \\ 
    \hline 
    Korrekturfaktoren               & $\alpha_1 = \alpha_2 = 2$                                 & $\alpha_1 \approx \alpha_2 \approx 1$                     \\ 
    \hline 
    Druckverlust $\Delta \, p_v$    & \multicolumn{2}{c}{$\Delta p_v = \lambda_x \frac{l}{d} \cdot \frac{\rho}{2} \cdot v^2$}                               \\ 
    \hline 
                                    & $\lambda_{\rm laminar} = \frac{64}{\Rey}$                 & $\lambda_{\rm turbulent} = \frac{0.316}{\sqrt[4]{\Rey}}$  \\
\end{ctabular} 
\renewcommand{\arraystretch}{1}


\subsection[Druckwiderstand]{Druckwiderstand $\bm{F_D}$}

Beschreibt die turbulente Luftreibungskraft $F_R$ und wird meist als Luftwiderstand bezeichnet.

$$ \boxed{ F_D = \Delta \, p \cdot A_s = \frac{1}{2} \, \cdot \rho \cdot v^2 \cdot A_s \cdot c_W } $$

\renewcommand{\arraystretch}{1.3}
\begin{tabular}{c l c}
    $F_D$           & Druckwiderstand                           & $[F_D] = \newton$                     \\
    $\Delta \, p$   & Druckdifferenz                            & $[\Delta \, p] = \pascal$             \\
    $\rho$          & Luft-Dichte                               & $[\rho] = \frac{\kilogram}{\meter^3}$ \\
    $c_W$           & Widerstandsbeiwert / Widerstandszahl      & $[c_W] = 1$                           \\
    $v$             & Strömungs-Geschwindigkeit                 & $[v] = \frac{\meter}{\second}$        \\
    $A_s$           & Projizierte Fläche senkrecht zur Strömung & $[A_s] = \meter$
\end{tabular}
\renewcommand{\arraystretch}{1}

\medskip

\textrightarrow\ Der Widerstandsbeiwert $c_W$ ist \textbf{geometrieabhängig}!



\subsection[Auftriebskraft nach Kutta-Jukowski]{Auftriebskraft $\bm{F_A}$ nach Kutta-Jukowski}

Beschreibt die Proportionalität zwischen dynamischem Auftrieb und Zirkulation.

$$ \boxed{ F_A = \rho \cdot v \cdot l \cdot \Gamma } $$

\renewcommand{\arraystretch}{1.3}
\begin{tabular}{c l c}
    $F_A$       & Dynamischer Auftrieb      & $[F_A] = \newton$                     \\
    $\rho$      & Dichte des Fluids         & $[\rho] = \frac{\kilogram}{\meter^3}$ \\
    $v$         & Geschwindigkeit           & $[v] = \frac{\meter}{\second}$        \\
    $l$         & Länge quer zur Strömung   & $[l] = \meter$                        \\
    $\Gamma$    & Zirkulation               & $[\Gamma] = \frac{\meter^2}{\second}$
\end{tabular}
\renewcommand{\arraystretch}{1}


\subsubsection{Zirkulation $\bm{\Gamma}$}

Die Zirkulation ist ein Mass für die \textbf{Rotation} im Strömungsfeld

$$ \boxed{ \Gamma = \oint \vec{v} \bullet \diff \vec{s} } $$

\renewcommand{\arraystretch}{1.5}
\begin{tabular}{c l c}
    $\Gamma$                        & Zirkulation                       & $[\Gamma] = \frac{\meter^2}{\second}$     \\
    $\vec{v} \bullet \iff \vec{s}$  & Geschwindigkeit entlang dem Weg   & $[\vec{v}] = \frac{\meter}{\second}$      \\
                                    & Skalarprodukt: $\vec{v} \bullet  \diff \vec{s} = a \cdot b \cdot \cos(\varphi)$
\end{tabular}
\renewcommand{\arraystretch}{1}


\subsection[Dynamischer Auftrieb]{Dynamischer Auftrieb $\bm{F_A}$}

\vspace{-0.2cm}

$$ \boxed{ F_A = c_A \cdot \underbrace{\frac{1}{2} \cdot \rho \cdot v^2 }_{\substack{\Delta \, p}} \cdot A_{\|}	} $$

\renewcommand{\arraystretch}{1.3}
\begin{tabular}{c l c}
    $F_A$       & Dynamischer Auftrieb                              & $[F_A] = \newton$                     \\
    $c_A$       & Auftriebskoeffizient                              & $[c_A] = 1$                           \\
    $\rho$      & Luft-Dichte                                       & $[\rho] = \frac{\kilogram}{\meter^3}$ \\
    $v$         & Strömungsgeschwindigkeit                          & $[v] = \frac{\meter}{\second}$        \\
    $A_{\|}$    & Projizierte Fläche \textbf{parallel} zur Strömung & $[A_{\|}] = \meter^2$
\end{tabular}
\renewcommand{\arraystretch}{1}


\subsubsection{Wissenswertes zum dynamischen Auftrieb}

\begin{itemize}
    \item Ein gerade ausgerichtetes, symmetrisches Stromlinienprofil erzeugt \textbf{keinen} dynamischen Auftrieb
    \item An einem asymmetrischen Flügelprofil entsteht dynamischer Auftrieb 
\end{itemize}


\subsection[Induzierter Widerstand]{Induzierter Widerstand $\bm{F_W}$}

Kommt durch Energieverlust (Wirbelbildung) zu Stande, welcher entsteht, wenn die Umgebungsluft in Bewegung gesetzt wird.

$$ \boxed{ F_W = c^*_W \cdot \frac{1}{2} \cdot \rho \cdot v^2 \cdot A_{\|} } $$


\begin{tabular}{c l c}
    $F_W$       & Induzierter Widerstand                            & $[F_W] = \newton$                     \\
    $\rho$      & Luft-Dichte                                       & $[\rho] = \frac{\kilogram}{\meter^3}$ \\
    $c^*_W$     & Widerstands-Koeffizient                           & $[c*_W] = 1$                          \\
    $v$         & Strömungsgeschwindigkeit                          & $[v] = frac{\meter}{\second}$         \\
    $A_{\|}$    & Projizierte Fläche \textbf{parallel} zur Strömung & $[A_{\|}] = \meter^2$
\end{tabular}

\columnbreak



\subsection[Gleitwinkel]{Gleitwinkel $\bm{\varphi}$}

Gibt die zurückgelegte Stecke pro verbrauchte Höhe an. Im Luft-Kanal ist dies der Anstell-Winkel.

$$ \boxed{ \tan(\varphi) = \frac{F_W}{F_A} = \frac{c^*_W}{c_A}= \frac{v_V}{v_H} } $$

\renewcommand{\arraystretch}{1.3}
\begin{tabular}{c l c}
    $\varphi$   & Gleitwinkel                   & $[\varphi] = \degree$             \\
    $F_W$       & Widerstandskraft              & $[F_W] = \newton$                 \\
    $F_A$       & Auftriebskraft                & $[F_A] = \newton$                 \\ 
    $c^*_W$     & Widerstands-Koeffizient       & $[c^*_W] = 1$                     \\
    $c_A$       & Auftriebs-Koeffizient         & $[c_A] = 1$                       \\
    $v_V$       & Vertikal-Geschwindigkeit      & $[v_V] = \frac{\meter}{\second}$  \\
    $v_H$       & Horizontal-Geschwindigkeit    & $[v_H] = \frac{\meter}{\second}$
\end{tabular}
\renewcommand{\arraystretch}{1}


\subsection{Helmholz'sche Wirbelsätze}

\begin{enumerate}
    \item Wirbel hat kein Anfang und kein Ende
    \item Wirbel besteht immer aus denselben Fluidteilchen
    \item Zirkulation zeitlich konstant
\end{enumerate}

\columnbreak



		\part{Thermodynamik}
		\section{Temperatur -- Dehnung / Streckung}

\subsection[Absolute Temperatur]{Absolute Temperatur $\bm{T}$}

\vspace{-0.2cm}

$$ \boxed{ T = \theta + 273.15 \, \kelvin = \theta - \theta_0 } $$

\begin{tabular}{c l c}
		$T$ 		& Absolute Temperatur \textbf{gemessen in Kelvin}				& $[T] =\kelvin$  		\\
		$\theta$ 	& Temperatur gemessen in \celsius 								& $[\theta] = \celsius$ \\
		$\theta_0$ 	& Absoluter Nullpunkt: $= -273.15 \, \celsius = 0 \, \kelvin$	& $[\theta_0] = \kelvin$
\end{tabular}


\subsection{Thermische Ausdehnung}

\subsubsection{Längenausdehnung $\bm{\Delta \, l}$}

\vspace{-0.2cm}

$$ \boxed{ l' = l + \Delta \, l = l + \alpha \cdot l \cdot \Delta \, T = l \, (1 + \alpha \cdot \Delta \, T ) } $$


\begin{tabular}{c l c}
		$l'$ 			& Länge nach Ausdehnung 		& $[l'] = \meter$ 					\\
		$l$ 			& Anfangslänge 					& $[l] = \meter$ 					\\
		$\Delta \, l$ 	& Längenänderung 				& $[\Delta \, l] = \meter$ 			\\
		$\alpha$ 		& Längenausdehnungskoeffizient 	& $[\alpha] = \frac{1}{\kelvin}$ 	\\ 
		$\Delta \, T $ 	& Temperaturänderung 			& $[\Delta \, T ] = \kelvin$
\end{tabular}


\subsubsection{Flächenausdehnung $\bm{\Delta \, A}$}

\vspace{-0.2cm}

$$ \boxed{ A' = A + \Delta \,  A = A + \underbrace{  \beta }_{\substack{\approx 2 \, \alpha}}  \cdot A \cdot \Delta \, T = A \, (1 + \beta \cdot \Delta \, T ) } $$


\begin{tabular}{c l c}
		$A'$ 			& Länge nach Ausdehnung 		& $[A'] = \meter^2$ 			\\
		$A$ 			& Anfangslänge 					& $[A] = \meter^2$ 				\\
		$\Delta \, A$	& Längenänderung 				& $[\Delta \, A] = \meter^2$ 	\\
		$\beta$ 		& Flächenausdehnungskoeffizient & $[\beta] = \frac{1}{\kelvin}$ \\ 
		$\Delta \, T $ 	& Temperaturänderung 			& $[\Delta \, T ] = \kelvin$
\end{tabular}


\subsubsection{Volumenausdehnung $\bm{\Delta \, V}$}

\vspace{-0.2cm}

$$ \boxed{ V' = V + \Delta \,  V = V + \underbrace{  \gamma }_{\substack{\approx 3 \, \alpha}}  \cdot V \cdot \Delta \, T = V \, (1 + \gamma \cdot \Delta \, T ) }$$


\begin{tabular}{c l c}
		$V'$ 			& Volumen nach Ausdehnung 		& $[A'] = \meter^3$ 			\\
		$V$ 			& Anfangsvolumen 				& $[A] = \meter^3$ 				\\
		$\Delta \, V$ 	& Volumenänderung 				& $[\Delta \, V] = \meter^3$ 	\\
		$\gamma$ 		& Volumenausdehnungskoeffizient & $[\beta] = \frac{1}{\kelvin}$ \\ 
		$\Delta \, T$ 	& Temperaturänderung 			& $[\Delta \, T ] = \kelvin$
\end{tabular}


\subsection[Thermische Spannung]{Thermische Spannung $\bm{\sigma}$}

\vspace{-0.2cm}

$$\boxed{ p = \sigma = \varepsilon \cdot E = E \cdot \frac{\Delta l}{l} =  E \cdot \alpha \cdot \Delta \, T } $$

\renewcommand{\arraystretch}{1.3}
\begin{tabular}{c l c}
	$\sigma$ 		& Thermische Spannung 			& $[\sigma] = \pascal$ 				\\
	$\varepsilon$ 	& Dehnung 						& $[\varepsilon] = 1$ 				\\
	$E$ 			& Elastizitätsmodul 			& $[E] = \frac{\newton}{\meter^2}$ 	\\
	$\alpha$ 		& Längenausdehnungskoeffizient 	& $[\alpha] = \frac{1}{\kelvin}$ 	\\ 
	$\Delta \, T $ 	& Temperaturänderung 			& $[\Delta \, T ] = \kelvin$ 		\\
	$p$ 			& Druck 						& $[p] = \pascal$
\end{tabular}
\renewcommand{\arraystretch}{1}


		
\section{Ideales Gas}

\subsection{Modell des idealen Gases}

\textbf{Jedes Gas ist gleich!}

\medskip

\begin{enumerate}
	\item Moleküle sind Massepunkte (keine Ausdehnung)
	\item Stösse sind elastisch (keine zwischenmolekularen Kräfte) \\
		Kein Volumen bei $T = 0$ \\
		Kein Druck bei $T = 0$
\end{enumerate}


\subsubsection{Thermische Ausdehnung von Gasen}

\begin{itemize}
	\item Ausdehnung von Gasen ist sehr gross
	\item Bei \textbf{allen} Gasen ist die Ausdehnung \textbf{gleich}
	\item Volumen beim Nullpunkt ist \textbf{Null} 
\end{itemize}


\subsection{Universelle Gasgleichung}

Alle Gase verhalten sich gleich, insbesondere bei gleicher Anzahl Moleküle

$$ \boxed{ \frac{p \cdot V}{T} = \, \const \quad \Leftrightarrow \quad \frac{p_1 \cdot V_1}{T_1} = \frac{p_2 \cdot V_2}{T_2} } $$ 

\begin{tabular}{c l c}
	$p_x$ & \textbf{Absolut-}Druck $p_0 + p$ 	& $[p_x] = \pascal$ 	\\
	$V_x$ & Volumen 							& $[V_x] = \meter^3$ 	\\
	$T_x$ & \textbf{Absolut-}Temperatur			& $[T] = \kelvin$
\end{tabular}
	
	
\subsubsection{Boyle-Mariotte}	

\textbf{Das Gesetz gilt nur bei konstanter Temperatur!} \\
\textrightarrow\ \textbf{isotherme} Zustandsänderung

$$  \boxed{ p \cdot V = \, \const \quad \Leftrightarrow \quad p_1 \cdot V_1 = p_2 \cdot V_2 } $$ 


\subsubsection{Gay-Lussac}

\textbf{ Das Gesetz gilt nur bei konstantem Druck!} \\
\textrightarrow\ \textbf{isobare} Zustandsänderung

$$  \boxed{ \frac{V}{T} = \, \const \quad \Leftrightarrow \quad \frac{V_1}{T_1} = \frac{V_2}{T_2} } $$

	
\subsubsection{Gay-Lussac und Amontons}

\textbf{Das Gesetz gilt nur bei konstantem Volumen!} \\
\textrightarrow\  \textbf{isochore} Zustandsänderung

$$ \boxed{ \frac{p}{T} = \, \const \quad \Leftrightarrow \quad \frac{p_1}{T_1} = \frac{p_2}{T_2} } $$
	
	
\subsection{Universelle Gasgleichung für ideale Gase}

\vspace{-0.2cm}

$$ \boxed{ p \cdot V = n \cdot R \cdot T = N \cdot k \cdot T }$$
	
\renewcommand{\arraystretch}{1.3}
\begin{tabular}{c l c}
	$p$	& \textbf{Absolut-}Druck $p_0 + p$ 											& $[p] = \pascal$ 							\\
	$V$	& Volumen 																	& $[V] = \meter^3$ 							\\
	$n$	& Mol-Zahl 																	& $[n] = \mole$	 							\\
	$R$	& Universelle Gaskonstante $R = 8.314 \, \frac{\joule}{\mole \, \kelvin}$ 	& $[R] = \frac{\joule}{\mole \, \kelvin}$ 	\\
	$T$	& \textbf{Absolut-}Temperatur 												& $[T] = \kelvin$ 							\\
	$N$	& Anzahl Moleküle 															& $[N] = 1$ 								\\
	$k$	& Boltzmann-Konstante $k = 1.381 \cdot 10^{-23} \, \frac{\joule}{\kelvin}$ 	& $[k] = \frac{\joule}{\kelvin}$
\end{tabular}
\renewcommand{\arraystretch}{1}
	
	
\subsubsection{Zusammenhänge zwischen den Konstanten}
	
\vspace{-0.2cm}

$$ \boxed{ R = k \cdot N_A = \frac{N \cdot k}{n} } \qquad \qquad \boxed{ n = \frac{N}{N_A} = \frac{m}{M} = \frac{N \cdot k}{R} }$$
	

\renewcommand{\arraystretch}{1.3}
\begin{tabular}{c l c}
	$R$		& Universelle Gaskonstante $R = 8.314 \, \frac{\joule}{\mole \, \kelvin}$ 	& $[R] = \frac{\joule}{\mole \, \kelvin}$ 	\\
	$k$		& Boltzmann-Konstante $k = 1.381 \cdot 10^{-23} \, \frac{\joule}{\kelvin}$ 	& $[k] = \frac{\joule}{\kelvin}$			\\
	$N$ 	& Anzahl Moleküle 															& $[N] = 1$ 								\\
	$N_A$ 	& Avogadrokonstante: $N_A = 6.022 \cdot 10^{23} \, \frac{1}{\mole}$ 		& $[N_A] = \frac{1}{\mole}$					\\	
	$n$ 	& Mol-Zahl 																	& $[n] = \mole$ 							\\
	$m$ 	& Masse 																	& $[m] = \kilogram$ 						\\
	$M$ 	& Mol-Masse 																& $[M] = \frac{\kilogram}{\mole}$
\end{tabular}
\renewcommand{\arraystretch}{1}



\subsection[Mechanische Arbeit von Gasen]{Mechanische Arbeit $\bm{\Delta W}$ von Gasen}
\label{MechArbeit}

Folgende Formel ist für Flüssigkeiten \textbf{nicht} gültig, da diese inkompressibel sind ($\Delta V = 0$)

$$ \boxed{ \Delta W = F \cdot \Delta s = p \cdot A \cdot \Delta s = p \cdot \Delta V } $$


\begin{tabular}{c l c}
	$\Delta W$ 	& Mechanische Arbeit von Gas	& $[\Delta W] = \joule$		\\
	$F$ 		& Kraft 						& $[F] = \newton$ 			\\
	$\Delta s$ 	& Wegänderung 					& $[\Delta s] = \meter$ 	\\
	$p$ 		& Druck 						& $[p] = \pascal$ 			\\
	$A$ 		& Fläche 						& $[A] = \meter^2$ 			\\
	$\Delta V$ 	& Volumenänderung 				& $[\Delta V] = \meter^3$
\end{tabular}


\subsection{Gesetz von Avogadro}

Ein Mol eines Gases nimmt bei Normalbedingungen immer das gleiche Volumen ein (=Molvolumen) 

\medskip

Ideale Gase enthalten bei gleichem Druck $p$ und gleicher Temperatur $T$ immer gleich viele Moleküle (im Molvolumen)



\subsection[Molmasse / Molvolumen]{Molmasse $\bm{M}$, Molvolumen $\bm{V_m}$}

Für 1 Mol Teilchen gilt: 

$$ \boxed{ p \cdot V = R \cdot T = N_A \cdot k \cdot T } $$

\medskip

\begin{minipage}[t]{0.58\columnwidth}
	\subsubsection{Molmasse}

	Entspricht der \textbf{Ordnungszahl} im Periodensystem
	$$  \boxed{ n = \frac{m}{M} = \frac{N}{N_A} } $$
\end{minipage}
\hfill
\begin{minipage}[t]{0.38\columnwidth}
	\subsubsection{Molvolumen}
	
	\phantom{eifach öppis}
	$$ \boxed{ V_m = \frac{V}{n} }$$
\end{minipage}


\renewcommand{\arraystretch}{1.3}
\begin{tabular}{c l c}
	$p$ 	& \textbf{Absolut-}Druck $p_0 + p$ 											& $[p] = \pascal$ 							\\
	$V$ 	& Volumen 																	& $[V] = \meter^3$ 							\\
	$R$		& Universelle Gaskonstante $R = 8.314 \, \frac{\joule}{\mole \, \kelvin}$ 	& $[R] = \frac{\joule}{\mole \, \kelvin}$ 	\\
	$T$ 	& \textbf{Absolut-}Temperatur 												& $[T] = \kelvin$ 							\\
	$N_A$	& Avogadrokonstante: $N_A = 6.022 \cdot 10^{23} \, \frac{1}{\mole}$ 		& $[N_A] = \frac{1}{\mole}$					\\	
	$k$		& Boltzmann-Konstante $k = 1.381 \cdot 10^{-23} \, \frac{\joule}{\kelvin}$ 	& $[k] = \frac{\joule}{\kelvin}$			\\
	$n$ 	& Mol-Zahl 																	& $[n] = \mole$ 							\\
	$m$ 	& Masse 																	& $[m] = \kilogram$ 						\\
	$M$ 	& Mol-Masse 																& $[M] = \frac{\kilogram}{\mole}$			\\
	$N$ 	& Anzahl Moleküle 															& $[N] = 1$ 								\\
	$V_m$ 	& Mol-Volumen 																& $[V_m] = \frac{\meter^3}{\mole}$
\end{tabular}
\renewcommand{\arraystretch}{1}


\subsection[Dichte eines Gases]{Dichte eines Gases $\bm{\rho}$}

\vspace{-0.2cm}

$$ \boxed{ \rho = \frac{m}{V} = \frac{M}{V_m} = \frac{p \cdot M}{R \cdot T} }$$

\renewcommand{\arraystretch}{1.3}
\begin{tabular}{c l c}
	$\rho$ 	& Gas-Dichte 																		& $[\rho] = \frac{\kilogram}{\meter^3}$		\\
	$m$ 	& Masse 																			& $[m] = \kilogram$ 						\\
	$V$ 	& Volumen 																			& $[V] = \meter^3$ 							\\
	$M$ 	& Mol-Masse 																		& $[M] = \frac{\kilogram}{\mole}$			\\
	$V_m$ 	& Mol-Volumen ($22.4 \, \liter$ bei $0 \, \celsius$ und $1000 \, \hecto \pascal)$	& $[V_m] = \frac{\meter^3}{\mole}$ 			\\
	$p$ 	& \textbf{Absolut-}Druck $p_0 + p$ 													& $[p] = \pascal$ 							\\
	$R$		& Universelle Gaskonstante $R = 8.314 \, \frac{\joule}{\mole \, \kelvin}$ 			& $[R] = \frac{\joule}{\mole \, \kelvin}$ 	\\
	$T$ 	& \textbf{Absolut-}Temperatur 														& $[T] = \kelvin$
\end{tabular}
\renewcommand{\arraystretch}{1}


\subsection{Phänomene von idealen Gasen}

\subsubsection{Annomalie des Wassers}

Die feste Form (Eis) ist leichter als die flüssige Form (Wasser) \\
Die \textbf{grösste Dichte weist Wasser bei 4 °C} auf, nicht beim Gefrierpunkt von $0 \, \celsius$

\smallskip

\textrightarrow\ Ein See gefriert somit nur an der Oberfläche. Am Grund des Sees beträgt die Wassertemperatur $4 \, \celsius$ 


\subsubsection{Osmotischer Druck (Zelldruck)}

Grosse Moleküle innerhalb von vielen kleinen Molekülen in einer Flüssigkeit verhalten sich ähnlich wie die Moleküle eines idealen Gases, 
wenn die Flüssigkeit von einer für die Müleküle halb-durchlässigen (semi-permeabel) Membran umgeben ist.

$$ \text{Osmotischer Druck:} \quad p = \frac{n}{V} \cdot R \cdot T  \qquad \text{(ideale Gasgleichung)}$$


\subsection[Partialdruck]{Partialdruck $\bm{p_i}$}

\textbf{Ausgangslage: Gasgemisch (z.B. Luft: Sauerstoff-Stickstoff)}

\begin{minipage}[c]{0.48\columnwidth}
	\includegraphics[width=\columnwidth,]{images/partialdruck.png}
\end{minipage}
\hfill
\begin{minipage}[c]{0.48\columnwidth}
	Der Partialdruck $p_i$ ist der Druck, welcher die $i$-te Gaskomponete erzeugen würde, wenn ihr das gesamte Volumen zur 
	Verfügung stehen würde.
\end{minipage}



\subsection{Gesetz von Dalton}

In einem Gas ist die Summe der Partialdrücke $p_i$ gleich dem Gesamtdruck 

\begin{minipage}[c]{0.38\columnwidth}
	$$ \boxed{ \sum_{i=1}^n  p_i = p } $$
\end{minipage}
\hfill
\begin{minipage}[c]{0.58\columnwidth}
	\begin{tabular}{c l c}
		$p_i$	& Partialdruck 		& $[p_i] = \pascal$	\\
		$p$ 	& (Gesamt-) Druck 	& $[p] = \pascal$ 	
	\end{tabular}
\end{minipage}
 


\subsection{Volumen- und Massenkonzentration (Gasgemisch)}

\subsubsection{Volumen-Konzentrationen (Volumen-Anteile)}

\vspace{-0.2cm}

$$ \boxed{  q_i = \frac{V_i}{V} = \frac{n_i}{n} = \frac{p_i}{p} } $$


\begin{tabular}{c l c}
	$q_i$ 	& Volumen-Konzentration						& $[q_i] = 1$ 			\\
	$V_i$ 	& Volumen der $i$-ten Gas-Komponente 		& $[V_i] = \meter^3$ 	\\
	$V$ 	& Gesamt-Volumen	 						& $[V] = \meter^3$ 		\\
	$n_i$ 	& Molzahl der $i$-ten Gas-Komponente 		& $[n_i] = \mole$ 		\\
	$n$ 	& Gesamt-Molzahl des Gemischs 				& $[n] = \mole$ 		\\
	$p_i$ 	& Partialdruck der $i$-ten Gaskomponente 	& $[p_i] = \pascal$ 	\\
	$p$ 	& Druck des Gemischs 						& $[p] = \pascal$
\end{tabular}


\subsubsection{Massen-Konzentration (Massen-Anteile)}

\vspace{-0.2cm}

$$ \boxed{ \mu_i = \frac{m_i}{m} = \frac{M_i}{M} \cdot q_i } $$

\renewcommand{\arraystretch}{1.3}
\begin{tabular}{c l c}
	$\mu_i$ & Volumen-Konzentrationen 				& $[\mu_i] = 1$ 					\\
	$m_i$ 	& Masse der $i$-ten Gas-Komponente 		& $[m_i] = \kilogram$ 				\\
	$m$ 	& Masse der Gemischs 					& $[m] = \kilogram$ 				\\
	$M_i$ 	& Mol-Masse der $i$-ten Gas-Komponete 	& $[M_i] = \frac{\kilogram}{\mole}$ \\
	$M$ 	& Mol-Masse des Gemischs 				& $[M] = \frac{\kilogram}{\mole}$ 	\\
	$q_i$ 	& Volumen-Konzentration					& $[q_i] = 1$
\end{tabular}
\renewcommand{\arraystretch}{1}


\subsection{Mol-Masse Gasgemisch}

Die Mol-Masse des Gas-Gemischs kann als gewichteter Mittelwert berechnet werden, gewichtet mit den jeweiligen Volumen-Anteilen.

$$ \boxed{ M = \sum_{i=1}^n  q_i \cdot M_i } $$

\renewcommand{\arraystretch}{1.3}
\begin{tabular}{c l c}
	$M$ 	& Mol-Masse Gasgemisch								& $[M] = \frac{\kilogram}{\mole}$ 	\\
	$q_i$ 	& Volumen-Konzentration der $i$-ten Gas-Komponente	& $[q_i] = 1$ 						\\
	$M_i$ 	& Mol-Masse der $i$-ten Gas-Komponete 				& $[M_i] = \frac{\kilogram}{\mole}$
\end{tabular}
\renewcommand{\arraystretch}{1}


		\section{Reales Gas}

Im vergleich zum idealen Gas müssen zwei Dinge berücksichtigt werden:

\medskip

\begin{description}
	\item[Eigen-Volumen:] Ideales Gas hat \textbf{kleineres} Volumen als gemessen \\
		( \textrightarrow\ Ideal-Gas-Volumen um das Molekül-Eigenvolumen reduzieren)
	\item[Binnen-Druck:] Ideales Gas hat \textbf{grösseren} Druck als gemessen \\
		( \textrightarrow\ Ideal-Gas-Druck um Binnendruck erhöhen) 
\end{description}


\subsection{Van der Waals-Gleichung (1 mol)}

\textbf{\textrightarrow\ Für nicht-ideale Gase!} 

$$ \boxed{ p' \cdot V'_m = R \cdot T } \qquad \qquad \boxed{ p' = p + \frac{a}{V_m^2} } \qquad \qquad \boxed{ V'_m = V_m - b } $$


\renewcommand{\arraystretch}{1.3}
\begin{tabular}{c l c}
	$p'$ 	& Korrigierter Druck 														& $[p'] = \pascal$ 								\\
	$V'_m$ 	& Korrigiertes Mol-Volumen 													& $[V_m] = \frac{\meter^3}{\mole}$ 				\\
	$R$		& Universelle Gaskonstante $R = 8.314 \, \frac{\joule}{\mole \, \kelvin}$ 	& $[R] = \frac{\joule}{\mole \, \kelvin}$ 		\\
	$T$ 	& \textbf{Absolut-}Temperatur		 										& $[T] = \kelvin$ 								\\
	$p$ 	& Druck des Gemischs 														& $[p] = \pascal$ 								\\
	$a$ 	& Eigenvolumen 																& $[a] = \frac{\joule \, \meter^3}{\mole^2}$ 	\\
	$b$ 	& Binnendruck 																& $[b] = \frac{\meter^3}{\mole}$ 				\\
	$V_m$ 	& Mol-Volumen 																& $[V_m] = \frac{\meter^3}{\mole}$
\end{tabular}
\renewcommand{\arraystretch}{1}


\subsection[Van der Waals-Gleichung (n Mol)]{Van der Waals-Gleichung ($\bm{n}$ mol)}

\vspace{-0.2cm}

$$ \boxed{ \Big( p + \frac{n^2 \cdot a}{V^2} \Big)  \cdot (V - n \cdot b) = n \cdot R \cdot T } $$

\renewcommand{\arraystretch}{1.3}
\begin{tabular}{c l c}
	$p$ & Druck des Gemischs 															& $[p] = \pascal$ 								\\
	$n$ & Mol-Zahl 																		& $[n] = \mole$ 								\\
	$a$ & Eigenvolumen 																	& $[a] = \frac{\joule \, \meter^3}{\mole^2}$ 	\\
	$V$ & Volumen 																		& $[V] = \meter^3$ 								\\
	$b$ & Binnendruck 																	& $[b] = \frac{\meter^3}{\mole}$ 				\\
	$R$	& Universelle Gaskonstante $R = 8.314 \, \frac{\joule}{\mole \, \kelvin}$ 		& $[R] = \frac{\joule}{\mole \, \kelvin}$ 		\\
	$T$ & \textbf{Absolut-}Temperatur 													& $[T] = \kelvin$
\end{tabular}
\renewcommand{\arraystretch}{1}


\subsubsection{Van der Waals-Parameter}

\vspace{-0.2cm}
$$ \boxed{ a = \frac{9}{8} \cdot R \cdot T_k \cdot V_{mk} }  \qquad \qquad  \boxed{ b = \frac{V_{mk}}{3} }$$
$$ \boxed{ V_{mk} = 3 \cdot b } \quad \quad \boxed{ T_k = \frac{8 \cdot a}{27 \cdot R \cdot b} } \quad \quad  \boxed{ p_k = \frac{a}{27 \cdot b^2} } $$


\renewcommand{\arraystretch}{1.3}
\begin{tabular}{c l c}
	$a$ 		& Eigenvolumen 																	& $[a] = \frac{\joule \, \meter^3}{\mole^2}$ 	\\
	$R$			& Universelle Gaskonstante $R = 8.314 \, \frac{\joule}{\mole \, \kelvin}$ 		& $[R] = \frac{\joule}{\mole \, \kelvin}$ 		\\
	$T_k$ 		& Kritische \textbf{Absolut-} Temperatur 										& $[T_k] = \kelvin$ 							\\
	$V_{mk}$ 	& Kritisches Mol-Volumen 														& $[V_{mk}] = \frac{\meter^3}{\mole}$ 			\\
	$b$ 		& Binnendruck 																	& $[b] = \frac{\meter^3}{\mole}$ 				\\
	$p_k$ 		& Kritischer Druck 																& $[p_k] = \pascal$
\end{tabular}
\renewcommand{\arraystretch}{1}


		\section{Wärmelehre}

\subsection{Wärme Q}

Wärme ist Energie, welche stets \textbf{(von allein)} von höherer zu niederigerer Temperatur fliesst.



\begin{ctabular}{l c l}
 				& $\underleftarrow{1. \text{ HS } 100\%}$ 		& 							\\
	$\Delta U$ 	& $=$ 											& $\Delta W + \Delta Q$ 	\\
				& $\overrightarrow{2 \text{ HS } \xout{100\% }}$
\end{ctabular}


\subsection{Erster Hauptsatz der Wärmelehre}

Nicht nur durch Wärmezufuhr, sondern auch durch mechanische Arbeit lässt sich die Temperatur und damit die innere Energie $U$ erhöhen.

$$ \boxed{ \Delta U =\Delta W + \Delta Q } $$

\begin{tabular}{c l c}
	$\Delta U$ 	& Zu-/Abgeführte Innere Energie 															& $[\Delta U] = \joule$ \\
	$\Delta W$ 	& Zu-/Abgeführte Arbeit z.B. $E_{\rm kin}, \; E_{\rm pot}, \; W_{\rm Gas}, \; W_{\rm reib}$	& $[\Delta W] = \joule$ \\
	$\Delta Q$ 	& Zu-/Abgeführte Wärme 																		& $[\Delta Q] = \joule$
\end{tabular}


\subsubsection{Ansätze für 1. HS}

\vspace{-0.2cm}

$$ \Delta Q = E_{\rm kin} = \frac{1}{2} \, m \cdot v^2 \qquad \qquad \Delta Q = E_{\rm pot} = m \cdot g \cdot h \qquad \qquad \Delta \dot{Q} = \Delta P $$


\subsubsection{Mechanische Arbeit eines Gases}

Für mehr Details, siehe Abschnitt ~\ref{MechArbeit}

$$  \boxed{\Delta W = p \cdot \Delta V } $$


\subsection{Mechanische Wärmeäquivalente}

1 Kalorie = $4,1868 \, \joule$ (cal) \\
\textrightarrow\ Energie, um 1 Gramm Wasser um 1 Grad zu erwärmen

\medskip

1 kcal = $4186,8 \, \joule$ \\
\textrightarrow\ Energie, um 1 Kilogramm Wasser um 1 Grad zu erwärmen


\subsubsection{Elektrisches Wärmeäquivalent $\bm{c}$}

\textbf{Elektrische Energie = Wärme}

$$ \boxed{ U \cdot I \cdot t = c \cdot m \cdot \Delta T \quad \Leftrightarrow \quad c = \frac{U \cdot I \cdot t}{m \cdot \Delta T} } $$


\begin{tabular}{c l c}
	$c$ 		& Elektrisches Wärmeäquivalent	& $[c] = \frac{\joule}{\kilogram \, \kelvin}$		\\
	$U$ 		& Spannung			 			& $[U] = \volt$ 									\\
	$I$ 		& Strom 						& $[I] = \ampere$ 									\\
	$t$ 		& Zeit 							& $[t] = \second$			 						\\
	$m$ 		& Masse 						& $[m] = \kilogram$ 								\\
	$\Delta T$ 	& Temperaturänderung 			& $[\Delta T] = \kelvin$
\end{tabular}


\subsection{Wärmekapazität}

Die Wärmekapazität drückt das Energiespeicher-Vermögen aus.

$$ \boxed{ Q = c \cdot m \cdot \Delta T = n \cdot C_m \cdot \Delta T = C \cdot \Delta T } $$


\subsubsection{Absolute Wärmekapazität $\bm{C}$}

Energiespeicher-Vermögen eines \textbf{Gegenstands}

$$ \boxed{ \Delta Q = C \cdot \Delta T } $$


\subsubsection{Spezifische Wärmekapazität $\bm{c}$}

Energiespeicher-Vermögen einer \textbf{Substanz}

$$ \boxed{ \Delta Q = c \cdot m \cdot \Delta T } \qquad \qquad c_{\rm Wasser} = 4187 \frac{\joule}{\kilogram \, \kelvin} $$


\subsubsection{Molare Wärmekapazität $\bm {C_m}$}

Energiespeicher-Vermögen einer \textbf{Anzahl Moleküle}

$$ \boxed{ C_m = \frac{c}{n} = M \cdot c } $$

\renewcommand{\arraystretch}{1.3}
\begin{tabular}{c l c}
	$\Delta Q$	& Zu-/Abgeführte Wärme 			& $[\Delta Q] = \joule$ 						\\
	$c$ 		& Spezifische Wärmekapazität 	& $[c] = \frac{\joule}{\kilogram \, \kelvin}$ 	\\
	$C$ 		& Absolute Wärmekapazität 		& $[C] = \frac{\joule}{\kelvin}$ 				\\
	$C_m$ 		& Molare Wärmekapazität 		& $[C_m] = \frac{\joule}{\mole \, \kelvin}$ 	\\
	$m$ 		& Masse 						& $[m] = \kilogram$ 							\\
	$\Delta T$ 	& Temperaturänderung 			& $[\Delta T] = \kelvin$ 						\\
	$n$ 		& Mol-Zahl 						& $[n] = \mole$ 								\\
	$M$ 		& Mol-Masse 					& $[M] = \frac{\kilogram}{\mole}$
\end{tabular}
\renewcommand{\arraystretch}{1}


\subsubsection{Molare Wärmekapazität von Gasen}

\vspace{-0.2cm}

$$ \boxed{ C_{mp} - C_{mV} = R } $$

\renewcommand{\arraystretch}{1.3}
\begin{tabular}{c l c}
	$C_{mp}$ 	& Isobare Wärme-Kapazität $(p = \const)$ 									& $[C_{mp}] = \frac{\joule}{\mole \, \kelvin}$	\\
	$C_{mV}$ 	& Isochore Wärme-Kapazität $(V = \const)$ 									& $[C_{mV}] = \frac{\joule}{\mole \, \kelvin}$	\\
	$R$			& Universelle Gaskonstante $R = 8.314 \, \frac{\joule}{\mole \, \kelvin}$ 	& $[R] = \frac{\joule}{\mole \, \kelvin}$
\end{tabular}
\renewcommand{\arraystretch}{1}


\subsubsection{Molare Wärmekapazität von Festkörpern}

\vspace{-0.4cm}

\begin{align*}
	T > \Theta_D: 	\quad &C_m \approx 3 \, R \approx 25 \, \frac{\joule}{\mole}  &\text{ (Dulung-Petit)} \\
	T \ll \Theta_D:	\quad &C_m = \frac{12 \cdot \pi^4}{5}  \cdot R \cdot  \Bigg( \frac{T}{\Theta_D}  \Bigg)^3  &\text{ (Debye)} 
\end{align*}


\renewcommand{\arraystretch}{1.3}
\begin{tabular}{c l c}
	$T$ 		& \textbf{Absolut-}Temperatur 												& $[T] = \kelvin$ 							\\
	$\Theta_D$ 	& Debye-Temperatur $\Theta_D \approx 200 \, \kelvin$ 						& $[\Theta_D] = \kelvin$ 					\\
	$C_m$ 		& Molare Wärmekapazität 													& $[C_m] = \frac{\joule}{\mole \, \kelvin}$ \\
	$R$			& Universelle Gaskonstante $R = 8.314 \, \frac{\joule}{\mole \, \kelvin}$ 	& $[R] = \frac{\joule}{\mole \, \kelvin}$
\end{tabular}
\renewcommand{\arraystretch}{1}


\subsection{Latente Wärme (Schmelz-/ Verdampfungswärme)}

\begin{minipage}[t]{0.48\columnwidth}
	\includegraphics[width=\columnwidth, align=t]{images/latente_waerme_2.png}
\end{minipage}
\hfill
\begin{minipage}[t]{0.48\columnwidth}
	Beim Schmelzen und Verdampfen findet \textbf{keine} Temperaturerhöhung statt.

	\medskip

	Beim Gefrieren und oder Kondensieren wird diese versteckte Wärme wieder frei, \textbf{ohne} Abnahme der Temperatur
\end{minipage}

\medskip

\textbf{\textrightarrow\ Die Schmelz-/ Verdampfungswärme ist stark druckabhängig}


$$ \boxed{ Q_f = q_f \cdot m } \qquad \qquad q_{f_{\rm Wasser}} := 334  \, \frac{\kilo \joule}{\kilogram} $$
$$ \boxed{ Q_S = q_s \cdot m } \qquad \qquad q_{s_{\rm Wasser}} := 2256 \, \frac{\kilo \joule}{\kilogram} $$


\renewcommand{\arraystretch}{1.3}
\begin{tabular}{c l c}
	$Q_f$ 	& Schmelz-/Erstarrungs-Wärme	 	& $[Q_f] = \joule$ 						\\
	$q_f$ 	& Spezifische Schmelzwärme 			& $[q_f] = \frac{\joule}{\kilogram}$	\\
	$Q_S$ 	& Verdampfungs-/Kondensations-Wärme & $[Q_S] = \joule$ 						\\
	$q_s$ 	& Spezifische Verdampfungs-Wärme	& $[q_s] = \frac{\joule}{\kilogram}$	\\
	$m$ 	& Masse 							& $[m] = \kilogram$
\end{tabular}
\renewcommand{\arraystretch}{1}



\subsection{Wärmebilanz}

Wärmeaustausch zwischen verschiedenen Materialien

\smallskip

In einem abgeschlossenen System (nach aussen isoliert) muss gelten: \\
\textbf{Zugeführte Wärme = Abgeführte Wärme}

$$  \boxed{ \sum_{i=1}^n  ( \Delta Q_i + \Delta Q_{f_i} + \Delta Q_{s_i} ) = 0 } $$


\begin{tabular}{c l c@{}}
	$\Delta Q_i$		& $i$-te Wärme-Menge aus Temperatur-Zu-/Abnahme					& $[\Delta Q_i] = \joule$ 			\\
	$\Delta Q_{f_i}$ 	& $i$-te Wärme-Menge aus Schmelz-/Erstarrungs-Vorgang			& $[\Delta Q_{f_i}] = \joule$ 		\\
	$\Delta Q_{s_i}$ 	& $i$-te Wärme-Menge aus Verdampfungs-/Kondensations-Vorgang 	& $[\Delta Q_{s_i}] = \joule$		\\
	$+$					& Zugeführte Wärme-Menge  																			\\
	$-$					& Abgeführterr Wärme-Menge
\end{tabular}


		\section{Phasen und Phasenübergänge}

\subsection{Phasen}

\begin{description}
	\item[Fest:] feste Gestalt; festes Volumen
	\item[Flüssig:] keine feste Gestalt; festes Volumen
	\item[Gasförmig:] keine feste Gesalt; kein festes Volumen
	\item[Plasma:] bei sehr hoher Temperatur ist Materie ionisiert (Elektronengas)
\end{description}


\subsection[Dampfdruck]{Dampfdruck $\boldsymbol{p_s(T)}$}

\begin{itemize}
	\item \textbf{Der Dampfdruck bedeutet das Gleichgewicht der Flüssigkeit mit ihrer Dampfphase}
	\item Der Dampfdruck ist das Niveau des kontanten Drucks im 2-Phasengebiet eines realen Gases nach van der Waals
	\item Der Dampfdruck ist nur \textbf{temperaturabhängig}
	\item Bei Kompression oder Expansion ändert sich der Dampfdruck nicht, sondern der Anteil Flüssigkeit zu Gas muss ändern
\end{itemize}

\begin{center}
	\includegraphics[width=0.9\columnwidth]{images/dampfdruck.jpg}
\end{center}

\begin{itemize}
	\item \textbf{Verdunsten} \textrightarrow\ Schnellste Teilchen treten aus Flüssigkeit aus
	\item \textbf{Sieden/Verdampfen} \textrightarrow\ Dampfdruck = Umgebungsdruck
\end{itemize}



\subsection{Dampfdruck-Kurve (Clausius-Clapeyron)}

\textbf{Kondensieren \textlrarrow\ Verdampfen}  \qquad flüssig \textlrarrow\ gasförmig

$$ \boxed{ \frac{\diff \, p_s}{\diff \, T} = \frac{q_s}{T \cdot  \Big( \frac{1}{\rho_g} - \frac{1}{\rho_f} \Big) } } $$


\subsubsection{Dampfdruck $\boldsymbol{p_s(T)}$ von Wasser (Clausius-Clapeyron)}

\vspace{-0.2cm}

$$ \boxed{ p_s(T) = p_{s0} \cdot e^{\frac{q_s \cdot M_W}{R}} \cdot \Big( \frac{1}{T_0} - \frac{1}{T} \Big) } $$
$$ p_{s0} = 610.7 \, \pascal \qquad T_0 = 273 \, \kelvin \qquad q_s = 2420 \, \frac{\kilo \joule}{\kilogram} \qquad M_W = 18.02 \, \frac{\gram}{\mole} $$



\subsection{Schmelzdruck-Kurve (Clausius-Clapeyron)}

\textbf{Erstarren \textlrarrow\ Schmelzen}  \qquad fest \textlrarrow\ flüssig

$$ \boxed{ \frac{\diff \, p_f}{\diff \, T} = \frac{q_f}{T \cdot \Big( \frac{1}{p_f} - \frac{1}{p_s} \Big) } } $$


\subsection{Gasdruck-Kurve (Clausius-Clapeyron)}

\textbf{Desublimieren \textlrarrow\ Sublimieren} \qquad fest \textleftarrow\ gasförmig

$$ \boxed{ \frac{\diff \, p_{sub}}{\diff \, T} = \frac{q_s + q_f}{T \cdot \Big( \frac{1}{\rho_g} - \frac{1}{\rho_s} \Big) } } $$


\renewcommand{\arraystretch}{1.3}
\begin{tabular}{c l c}	
	$q_s$ 		& Spezifische Verdampfungs-Wärme 	& $[q_s] = \frac{\joule}{\kilogram}$		\\
	$q_f$ 		& Spezifische Schmelz-Wärme 		& $[q_f] = \frac{\joule}{\kilogram}$ 		\\
	$q_s + q_f$ & Spezifische Sublimations-Wärme 	&	 										\\
	$p_s$ 		& Dampfdruck 						& $[p_s] = \pascal$ 						\\
	$p_f$ 		& Schmelzdruck 						& $[p_f] = \pascal$ 						\\
	$p_g$ 		& Schmelzdruck 						& $[p_g] = \pascal$ 						\\
	$\rho_g$ 	& Dichte Gas 						& $[\rho_g] = \frac{\kilogram}{\meter^3}$	\\
	$\rho_f$ 	& Dichte Flüssgkeit 				& $[\rho_f] = \frac{\kilogram}{\meter^3}$	\\
	$\rho_s$ 	& DichteFestkörper 					& $[\rho_s] = \frac{\kilogram}{\meter^3}$	
\end{tabular}
\renewcommand{\arraystretch}{1}


\subsection{Formeln von Magnus}

Die Formeln von Magnus dienen der vereinfachten Berechnung des Dampfdrucks von Wasser = Sättigungsdruck 

\subsubsection{Dampfdruck von Wasser $\boldsymbol{p_s(\theta) \quad (\theta \geq 0 \, \celsius)}$}

\vspace{-0.2cm}

$$ \boxed{ p_s(\theta) = p_{s0} \cdot 10^{ \frac{7.5 \cdot \theta}{\theta + 237} } } $$



\subsubsection{Schmelzdruck von Wasser $\bm{p_s(\theta) \quad (\theta \leq 0 \, \celsius)}$}

\vspace{-0.2cm}

$$ \boxed{ p_s(\theta) = p_{s0} \cdot 10^{ \frac{9.5 \cdot \theta}{\theta + 265.5} } } $$


\begin{tabular}{c l c}
	$p_s$ 		& Dampfdruck / Schmelzdruck 									& $[p_s] = \pascal$ 	\\
	$p_{s0}$ 	& Dampfdruck bei $0\, \celsius \quad p_{s0} = 610.7 \, \pascal$	& $[p_{s0}] = \pascal$	\\
	$\theta$ 	& Temperatur 													& $[\theta] = \celsius$
	
\end{tabular}


\subsection{Umkehrformeln von Magnus}

\begin{minipage}[t]{0.48\columnwidth}
	\subsubsection{$\boldsymbol{\theta(p_s)}$ für $\boldsymbol{p_s \geq p_{s0}}$}

	\vspace{-0.2cm}

	$$ \boxed{ \theta(p_s) = \frac{237 \cdot \log \big( \frac{p_s}{6.107} \big) }{7.5 - \log \big( \frac{p_s}{6.107} \big)} } $$	
\end{minipage}
\hfill
\begin{minipage}[t]{0.48\columnwidth}
	\subsubsection{$\boldsymbol{\theta(p_s)}$ für $\boldsymbol{p_s \leq p_{s0}}$}

	\vspace{-0.2cm}

	$$ \boxed{ \theta(p_s) = \frac{265.5 \cdot \log \big( \frac{p_s}{p_{s0}} \big)  }{9.5 - \log \big( \frac{p_s}{p_{s0}} \big)} } $$
\end{minipage}


\subsection{Luftfeuchtigkeit}

\begin{minipage}[t]{0.48\columnwidth}
	\subsubsection{Abs.Luftfeuchtigkeit $\bm{f}$}

	\vspace{-0.2cm}

	$$ \boxed{ f = \frac{m_W}{V}  } $$
\end{minipage}
\hfill
\begin{minipage}[t]{0.48\columnwidth}
	\subsubsection{Rel. Luftfeuchtigkeit $\boldsymbol{f_r}$}

	\vspace{-0.2cm}

	$$ \boxed{ f_r = \frac{m_W}{m_S} = \frac{p_D}{p_S} = \frac{p_D}{p_S(\theta)} } $$
\end{minipage}

\medskip

\begin{tabular}{c l c}
	$f$ 		& Absolute Luftfeuchtigkeit 				& $[f] = 1$ 			\\
	$f_r$ 		& Relative Luftfeuchtigkeit 				& $[f_r] = 1$ 			\\
	$m_W$ 		& Masse Wasserdampf 						& $[m_W] = \kilogram$ 	\\
	$m_S$ 		& Masse Wasserdampf bei Sättigung 			& $[m_S] = \kilogram$ 	\\
	$V$ 		& Volumen 									& $[V] = \meter^3$ 		\\
	$p_D$ 		& Partialdruck Wasserdampf 					& $[p_D] = \pascal$ 	\\
	$p_S$ 		& Dampfdruck = Sättigungsdruck Wasserdampf 	& $[p_s] = \pascal$ 	\\
	$\theta$ 	& Temperatur 								& $[\theta] = \celsius$
\end{tabular}


\subsubsection{Feuchte vs. trockene Luft}

\textbf{Feuchte Luft ist leichter als trockene Luft!}

$$ \boxed{ \rho_F < \rho_T } \qquad (\text{da } M_W < M_L) $$

\renewcommand{\arraystretch}{1.3}
\begin{tabular}{c l c}
	$\rho_F$	& Dichte feuchte Luft 	& $[\rho_F] = \frac{\kilogram}{\meter^3}$	\\
	$\rho_T$	& Dichte trockene Luft 	& $[\rho_F] = \frac{\kilogram}{\meter^3}$	\\
	$M_W$ 		& Molmasse $H_2O$ 		& $[M_W] = \frac{\gram}{\mole}$ 			\\
	$M_S$ 		& Molmasse Luft 		& $[M_W] = \frac{\gram}{\mole}$
\end{tabular}
\renewcommand{\arraystretch}{1}



\subsection[Taupunkts-Temperatur]{Taupunkts-Temperatur $\boldsymbol{\theta_d}$}

Temperatur, bei welcher 100\% Luftfeuchtigkeit herrscht.

\smallskip

Wenn die Taupunkt-Temperatur \textbf{unterschritten} wird, dann kondensiert Wasser.

$$ \boxed{ \theta_d (\theta, f_r) = \frac{237 \cdot \Big( \log(f_r) + \frac{7.5 \cdot \theta}{\theta + 237}    \Big)}{7.5 - \Big( \log(f_r) + \frac{7.5 \cdot \theta }{\theta + 237} \Big) } } $$
$$ \boxed{ \theta_d (x) = \frac{237 \cdot x}{7.5 - x}  \qquad  \text{mit } \quad  x(\theta, f_r) = \log(f_r) + \frac{7.5 \cdot \theta}{\theta + 237} } $$


\begin{tabular}{c l c}
	$\theta_d$	& Taupunkts-Temperatur 		& $[\theta_d] = \celsius$ 	\\
	$f_r$ 		& Relative Luftfeuchtigkeit & $[f_r] = 1$ 				\\
	$\theta$ 	& Temperatur 				& $[\theta] = \celsius$
\end{tabular}


\subsection[Relative Innen-Feuchte]{Relative Innen-Feuchte $\boldsymbol{f_{ri}}$}

\vspace{-0.2cm}

$$ \boxed{ f_{ri} = \frac{p_s(\theta_a)}{p_s(\theta_i)} \cdot f_{ra} } $$

\begin{tabular}{c l c}
	$f_{ri}$ 		& Relative Feuchte im Inneren 		& $[f_{ri}] = 1$ 				\\
	$f_{ra}$ 		& Relative Feuchte der Aussenluft 	& $[f_{ra}] = 1$ 				\\
	$p_s(\theta_i)$ & Dampfdruck bei Innentemperatur 	& $[p_s(\theta_i)] = \pascal$ 	\\
	$p_s(\theta_a)$ & Dampfdruck bei Aussentemperatur 	&  $[p_s(\theta_a)] = \pascal$
\end{tabular}


		\input{sections/09_kinetische_gas_theorie.tex}
		\input{sections/10_temperaturstrahlung.tex}
		\section{Rückwandlung innerer Energie}

\subsection{Zweiter Hauptsatz der Wärmelehre}

\begin{itemize}
	\item Innere Energie kann \textbf{nicht zu 100 \%} in Arbeit umgesetzt werden \\
		\textrightarrow\ Carnot-Wirkungsgrad ist der theoretisch höchstmögliche
	\item Wärme kann niemals \myul{von selbst} von einem kälteren Ort zu einem wärmeren Ort fliessen (Clausius)
	\item Es gibt keine \myul{periodisch wirkende} Maschine, die nichts anderes bewirkt als Erzeugung mechanischer Arbeit und 
		Abkühlung eines Wärme-Reservoirs (Kelvin) \\
		\textrightarrow\ Es gibt kein Perpetuum mobile 2. Art
\end{itemize}


\subsection{Kreisprozess (reversibler Prozess)}

\begin{center}
	Anfangszustand = Endzustand
\end{center}


\begin{minipage}[t]{0.48\columnwidth}
	\myul{\textbf{Rechts}laufender Kreisprozess}

	\smallskip

	\begin{itemize}
		\item Gibt Arbeit ab
		\item \cbl{Wärmekraftmaschine}
		\item Bei hoher Temperatur wird Wärme aus Prozess \textbf{zu}geführt
		\item Nur Bruchteil der Wärme in Arbeit verwandelbar
		\item Obergrenze: Carnot-Wirkungsgrad
	\end{itemize}
\end{minipage}
\hfill
\begin{minipage}[t]{0.48\columnwidth}
	\myul{\textbf{Links}laufender Kreisprozess}

	\smallskip

	\begin{itemize}
		\item Verbraucht Arbeit
		\item \cvt{Wärmepumpe}
		\item Bei hoher Temperatur wird dem Prozess Wärme \textbf{ab}geführt
		\item Erzeugt mehrfaches an Wärme
		\item Obergrenze: Inv. Carnot-Wirkungsgrad
	\end{itemize}
\end{minipage}





\subsection{Carnot-Wirkungsgrad}

\vspace{-0.2cm}

$$ \boxed{\text{\cbl{Wärmekraftmaschine:}}  \quad n_C = \frac{W_{\rm ab}}{Q_{\rm zu}} = \frac{T_{\rm hoch} - T_{\rm tief}}{T_{\rm hoch}} } $$

$$ \boxed{\text{ \cvt{Wärmepumpe:}} \quad  n_{iC} = \frac{Q_{\rm zu}}{W_{\rm ab}} = \frac{T_{\rm hoch}}{T_{\rm hoch} - T_{\rm tief}} } $$


\begin{tabular}{c l c}
	$n_C$ 			& Carnot-Wirkungsgrad 				& $[n_C] = 1$ 					\\
	$n_{iC}$ 		& Inverset Carnot-Wirkungsgrad 		& $[n_{iC}] = 1$ 				\\
	$T_{\rm tief}$ 	& Temperatur des Warm-Reservoirs 	& $[T_{\rm tief}] = \kelvin$	\\
	$T_{\rm hoch}$ 	& Temperatur des Kalt-Reservoirs 	& $[T_{\rm hoch}] = \kelvin$ 	\\
	$Q_{\rm zu}$ 	& zugeführte Wärme 					& $[Q_{\rm zu}] = \joule$ 		\\ 
	$W_{\rm ab}$ 	& abgeführte Energie 				& $[W_{\rm ab}] = \joule$
\end{tabular}


\subsection{Adiabaten-Gleichung (Kreisprozess)}

Adiabate wird beschrieben im pV- / TV- / Tp-Diagramm

\medskip

\begin{minipage}{0.6\columnwidth}
	\includegraphics[width=\columnwidth]{images/kreisprozess.png}
\end{minipage}
\hfill
\begin{minipage}{0.38\columnwidth}
	$$ \boxed{ p \cdot V^\kappa  = \const } $$
	$$ \boxed{ T \cdot V^{\kappa -1} = \const } $$
	$$ \boxed{ T^\kappa \cdot p^{1-\kappa} = \const } $$
	$$ \kappa = \frac{C_{mp}}{C_{mV}} $$
	$$ \boxed{ C_{mp} - C_{mV} = R } $$
\end{minipage}

\medskip

\renewcommand{\arraystretch}{1.3}
\begin{tabular}{c l c}
	$C_{mp}$ 	& Molare Wärmekapazität @ $p = \const$ 										& $[C_{mp}] = \frac{\joule}{\mole \, \kelvin}$	\\
	$C_{mV}$ 	& Molare Wärmekapazität @ $V = \const$  									& $[C_{mV}] = \frac{\joule}{\mole \, \kelvin}$	\\
	$\kappa$ 	& Adiabaten-Exponent 														& $[\kappa] = 1$ 								\\
	$R$			& Universelle Gaskonstante $R = 8.314 \, \frac{\joule}{\mole \, \kelvin}$ 	& $[R] = \frac{\joule}{\mole \, \kelvin}$
\end{tabular}
\renewcommand{\arraystretch}{1}


\subsection{Kreisprozesse (Vorgänge)}


\begin{tabular}{lll}
	isotherme Expansion 		& liefert Wärme 	& benötigt Energie	\\
	isotherme Kompression 		& benötigt Wärme 	& liefert Energie 	\\
	\\
	adiabatische Expansion 		& liefert Arbeit 	& ohne Wärme 		\\
	adiabatische Kompression 	& benötigt Arbeit 	& ohne Wärme 		\\
	\\
	isochore Erwärmung 			& ohne Arbeit 		& benötigt Wärme 	\\
	isochore Abkühlung 			& ohne Arbeit 		& liefert Wärme
\end{tabular}


\subsection{Beispiel Kreisprozess}

\begin{minipage}[c]{0.48\columnwidth}
	\includegraphics[width=\columnwidth]{images/kreisprozess_2.png}
\end{minipage}
\hfill
\begin{minipage}[c]{0.48\columnwidth}
	\includegraphics[width=\columnwidth]{images/kreisprozess_3.png}
\end{minipage}


\subsection{Entropie-Zunahme}

\subsubsection{Definition der Entropie-Zunahme}

\vspace{-0.2cm}

$$ \Delta S = S_1 + S_2 = \int \frac{1}{T} \, \diff Q $$


\subsubsection{Boltzmann-Gleichung für Entropie-Zunahme}

\vspace{-0.2cm}

$$ \boxed{ \Delta S = k \cdot \ln(W) } $$


\renewcommand{\arraystretch}{1.3}
\begin{tabular}{c l c}
	$\Delta S$	& Entropie 																	& $[\Delta S] = \frac{\joule}{\kelvin}$	\\
	$k$ 		& Boltzmann-Konstante $k = 1.381 \cdot 10^{-23} \, \frac{\joule}{\kelvin}$	& $[k] = \frac{\joule}{\kelvin}$		\\
	$W$ 		& Wahrscheinlichkeit eines Zustands 										& $[W] = 1$
\end{tabular}
\renewcommand{\arraystretch}{1}


\subsubsection{Abgeschlossenes System}

\renewcommand{\arraystretch}{1.2}
\begin{tabular}{ll}
	$ \Delta S \geq 0$	& Entropie kann nur zunehmen in abgeschlossenem System	\\
	$ \Delta S > 0$ 	& Irreversibler Prozess 								\\
	$ \Delta S = 0$ 	& Reversibler Prozess
\end{tabular}
\renewcommand{\arraystretch}{1}

\columnbreak


		
		\part{Anhang}
		\section{Molmassen wichtiger Atome}

\renewcommand{\arraystretch}{1.3}
\begin{tabular}{c | c | c }
    \textbf{Symbol} & \textbf{Molekül}  & \textbf{Molmasse}                 \\
    \hline
    H               & Wasserstoff       & $1.008 \, \frac{\gram}{\mole}$    \\
    C               & Kohlenstoff       & $12.011 \, \frac{\gram}{\mole}$   \\
    N               & Stickstoff        & $14.007 \, \frac{\gram}{\mole}$   \\
    O               & Sauerstoff        & $15.999 \, \frac{\gram}{\mole}$   \\
    Al              & Aluminium         & $26.982 \, \frac{\gram}{\mole}$   \\
    Si              & Silicium          & $28.982 \, \frac{\gram}{\mole}$   \\
\end{tabular}
\renewcommand{\arraystretch}{1.3}


\section{Ansätze zu Aufgaben}

\subsection{Barometer}

\begin{minipage}[t]{0.4\columnwidth}
    \includegraphics[width=\columnwidth, align=t]{images/manometer.jpg}
\end{minipage}
\hfill
\begin{minipage}[t]{0.5\columnwidth}
    $$ \boxed{ p_1 = p_0 + \underbrace{ \rho_{Fl} \cdot g \cdot h}_{\substack{p_s}} } $$
 
    \begin{tabular}{ll}
        $p_1$   & Gemessener Druck  \\
        $p_0$   & Luftdruck         \\
        $p_s$   & Schweredruck
    \end{tabular}

    \medskip

    \textrightarrow\ Bernoulli \\
    \textrightarrow\ Kontinuität
\end{minipage}




\subsection{Pitotrohr}
Prandtl'sches Staurohr; Staudruckmesser \\
Zur Messung von Strümungsgeschwindigkeiten 

\begin{center}
    \includegraphics[width=0.73\columnwidth]{images/pitotrohr.png}
\end{center}


$$ \boxed{ \text{Bernoulli horizontal:} \quad \underbrace{p_1}_{\substack{p_L}} + \frac{1}{2} \, \rho_1 \cdot \underbrace{v_1^2}_{\substack{0}} = \underbrace{p_2}_{\substack{p_L - \Delta p}} + \frac{1}{2} \, \underbrace{\rho_2}_{\substack{\rho_L}} \cdot v_2^2} $$

$$ 0 = - \Delta p + \frac{1}{2} \, \rho_L \cdot v_2^2 \qquad \Rightarrow \Delta p =\frac{1}{2} \, \rho_L \cdot v_2^2 $$

$$ \text{Gleichsetzen:} \quad \Delta p = \rho_{Fl} \cdot g \cdot h $$


\subsection{Pumpe}

\vspace{-0.2cm}

$$ \boxed{ W = P \cdot t = F \cdot \Delta s = p \cdot A \cdot \Delta s = p \cdot \Delta V } \qquad  \qquad \boxed{ F = p \cdot A }  $$

$$ \boxed{ P =  \frac{W}{t} = \frac{p \cdot V}{t} = p  \cdot  \dot{V} } $$


\subsection{Bewegungen}

\vspace{-0.2cm}

$$ \boxed{ P = F \cdot v } \qquad \qquad \boxed{ E_{\rm kin} = \frac{1}{2} \, m \cdot v^2 } $$



\subsection{U-Rohr}

\begin{minipage}[t]{0.35\columnwidth}
\includegraphics[width=\columnwidth, align=t]{images/u-rohr.png}
\end{minipage}
\hfill
\begin{minipage}[t]{0.6\columnwidth}

    Ansatz: Druckgleichgewicht 

    $$ \boxed{ p_1 = p_2 } $$

    $$ \boxed{ \rho_1 \cdot g \cdot h_1 = \rho_2 \cdot g \cdot h_2 } $$
\end{minipage}


\subsection{Wasser mit Dampf erhitzen}

Ein Tasse mit $m_W = 200 \, \gram$ Wasser  mit einer Temperatur von $T_K = 20 \, \celsius$ wird an
der Wasserdampfdüse einer Kaffeemaschine mittels Wasserdampf erhitzt. \\
Der aus der Kaffeemaschine ausströmende Wasserdampf ist $T_H = 96 \, \celsius$ heiss. \\
Am Schluss haben Sie 10 \% mehr Wasser in der Tasse. (entspricht $m_D$)\\
Wie warm ist das Wasser nun?

$$ \text{Ansatz: 1. Hauptsatz} \quad \Delta Q_{\rm zu} = \Delta Q_{\rm ab} $$ 
$$ m_W \cdot c_W \, (T_M - T_K) = q_s \cdot m_D + m_D \cdot c_W \, (T_H - T_M) $$


\subsection{Eis in Wasser schmelzen}

In einem Gefäss beifinden sich $m_W = 1 \, \kilogram$ Wasser. \\
Dazu wird ein Eiswürfel von $m_E = 20 \, \gram$ gegeben. \\
Das Eis hat eine Temperatur von $T_E = -5 \, \celsius$ und das Wasser hat eine Temperatur $T_W$. Die Temperatur $T_0$ steht 
für $0 \, \celsius$ bzw. $275.15 \, \kelvin$ \\
Gesucht ist die Mischtemperatur $T_M$ 

$$ \Delta Q_{\rm ab} = \Delta Q_{\rm zu} $$
$$ m_W \cdot c_W \cdot (T_W - T_M) = m_E \cdot c_E \cdot (T_0 - T_E) + q_f \cdot m_E + m_E \cdot c_W \cdot (T_M - T_0) $$



    \end{layout}
\end{document}
