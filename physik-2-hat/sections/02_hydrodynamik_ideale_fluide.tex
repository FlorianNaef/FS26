\section{Hydrodynamik - Ideale Fluide}

\textbf{Ideale Fluide nehmen keine Scherkräfte auf (keine Reibung) und sind inkompressibel.}


\subsection{Stromlinien-Modell}

\begin{itemize}
	\item Stromlinien zeigen Geschwindigkeit des Fluids
	\item \textbf{Dichte} Stromlinien bedeutet \textbf{hohe} Geschwindigkeit
	\item \textbf{Dünne} Stromlinien bedeutet \textbf{niedrige} Geschwindigkeit 
	\item Stationär: Stromlinien schneiden sich nicht 
\end{itemize}


\subsection{Kontinuitätsgleichung}

\begin{center}
	\includegraphics[width=0.7\columnwidth]{images/Kontinuitaet.png}
\end{center}

\vspace{-0.5cm}

$$ \boxed{ \frac{\Delta V}{\Delta t} = \dot{V} = A \cdot v = \const  \quad \Leftrightarrow \quad  A_1 \cdot v_1 = A_2 \cdot v_2 = \frac{\Delta V}{\Delta t} = \dot{V}} $$ 

\renewcommand{\arraystretch}{1.3}
\begin{tabular}{c l c}
	$\Delta V$	& Volumenänderung 					& $[\Delta V] = \meter^3$					\\
	$\Delta t$ 	& Zeitänderung 						& $[\Delta t] = \second$  					\\
	$\dot{V}$ 	& Volumenstrom (Volumen pro Zeit) 	& $[\dot{V}] = \frac{\meter^3}{\second}$	\\
	$A_x$ 		& Querschnittsfläche 				& $[A_x] = \meter^2$ 						\\
	$v_x$ 		& Geschwindigkeit der Flüssigkeit 	& $[v_x] = \frac{\meter}{\second}$
\end{tabular}
\renewcommand{\arraystretch}{1}

\medskip

\textrightarrow\ Gilt auch für Gase, wenn $v \ll v_{\rm Schall}$


\subsection{Bernoulli-Gleichung}

\begin{minipage}[c]{0.38\columnwidth}
	\raggedright
	Die Bernoulli-Gleichung beschreibt ein \myul{bewegtes} Fluid

	$$ \underbrace{ p + \rho \cdot g \cdot h }_{\substack{\mathrm{statisch}}} 
	+ \underbrace{ \frac{1}{2} \, \rho \cdot v^2 }_{\substack{\mathrm{dynamisch}}} = \const $$
\end{minipage}
\hfill
\begin{minipage}[c]{0.6\columnwidth}
	\includegraphics[width=\columnwidth]{images/Bernoulli.png}
\end{minipage}

\smallskip

$$ \boxed{ p_1 +  \rho \cdot g \cdot h_1 + \frac{1}{2} \, \rho \cdot v_1^2 = p_2 +  \rho \cdot g \cdot h_2 + \frac{1}{2} \, \rho \cdot v_2^2 } $$

\medskip


\begin{minipage}[t]{0.48\columnwidth}
	\subsubsection{Spezialfall: Horizontal}

	$$ \boxed{ p + \frac{1}{2} \, \rho \cdot v^2 = \const } $$
\end{minipage}
\hfill
\begin{minipage}[t]{0.48\columnwidth}
	\subsubsection{Spezialfall: Statik}

	$$ \boxed{ p + \rho \, \cdot g \cdot h =  \const} $$
\end{minipage}

\medskip


\subsubsection{Hydrodynamisches Paradoxon}

\textbf{Je grösser die Strömungsgeschwindigkeit, desto kleiner der Druck} \\
\textrightarrow\ Gegen jede Intuition!




\subsection{Bernoulli-Gleichung und Energieerhaltung}

Die in der Bernoulli-Gleichung vorkommenden Terme können als \myul{Energie pro Volumen} betrachtet werden.

\vspace{-0.3cm}

\begin{align*}
	E_{\rm Mech}	&= \text{Elastische Energie } + \text{ pot. Energie } + \text{ kin. Energie} \\
					&= p \cdot V + m \cdot g \cdot h + \frac{1}{2} \, m \cdot v^2 = \const
\end{align*}



Wenn durch das Volumen dividiert wird erhält man: 

\vspace{-0.3cm}

\begin{align*}
	\frac{E_{\rm Mech}}{\text{Volumen}}	&= \frac{\text{elatische Energie}}{\text{Volumen}} + \frac{\text{pot. Energie}}{\text{Volumen}} + \frac{\text{kin. Energie}}{\text{Volumen}} \\
										&= p + \rho \cdot g \cdot h + \frac{1}{2} \, \rho \cdot v^2 = \const
\end{align*}


Bei einer horizontalen Strömung entfällt die pot. Energie (pro Volumen)

\vspace{-0.3cm}

\begin{align*}
	\frac{E_{\rm Mech}}{\text{Volumen}}	&= \frac{\text{elatische Energie}}{\text{Volumen}} + \frac{\text{kin. Energie}}{\text{Volumen}} \\
										&= p + \frac{1}{2} \, \rho \cdot v^2 = \const
\end{align*}


\columnbreak

