\section{Temperatur -- Dehnung / Streckung}

\subsection[Absolute Temperatur]{Absolute Temperatur $\bm{T}$}

\vspace{-0.2cm}

$$ \boxed{ T = \theta + 273.15 \, \kelvin = \theta - \theta_0 } $$

\begin{tabular}{c l c}
		$T$ 		& Absolute Temperatur \textbf{gemessen in Kelvin}				& $[T] =\kelvin$  		\\
		$\theta$ 	& Temperatur gemessen in \celsius 								& $[\theta] = \celsius$ \\
		$\theta_0$ 	& Absoluter Nullpunkt: $= -273.15 \, \celsius = 0 \, \kelvin$	& $[\theta_0] = \kelvin$
\end{tabular}


\subsection{Thermische Ausdehnung}

\subsubsection{Längenausdehnung $\bm{\Delta \, l}$}

\vspace{-0.2cm}

$$ \boxed{ l' = l + \Delta \, l = l + \alpha \cdot l \cdot \Delta \, T = l \, (1 + \alpha \cdot \Delta \, T ) } $$


\begin{tabular}{c l c}
		$l'$ 			& Länge nach Ausdehnung 		& $[l'] = \meter$ 					\\
		$l$ 			& Anfangslänge 					& $[l] = \meter$ 					\\
		$\Delta \, l$ 	& Längenänderung 				& $[\Delta \, l] = \meter$ 			\\
		$\alpha$ 		& Längenausdehnungskoeffizient 	& $[\alpha] = \frac{1}{\kelvin}$ 	\\ 
		$\Delta \, T $ 	& Temperaturänderung 			& $[\Delta \, T ] = \kelvin$
\end{tabular}


\subsubsection{Flächenausdehnung $\bm{\Delta \, A}$}

\vspace{-0.2cm}

$$ \boxed{ A' = A + \Delta \,  A = A + \underbrace{  \beta }_{\substack{\approx 2 \, \alpha}}  \cdot A \cdot \Delta \, T = A \, (1 + \beta \cdot \Delta \, T ) } $$


\begin{tabular}{c l c}
		$A'$ 			& Länge nach Ausdehnung 		& $[A'] = \meter^2$ 			\\
		$A$ 			& Anfangslänge 					& $[A] = \meter^2$ 				\\
		$\Delta \, A$	& Längenänderung 				& $[\Delta \, A] = \meter^2$ 	\\
		$\beta$ 		& Flächenausdehnungskoeffizient & $[\beta] = \frac{1}{\kelvin}$ \\ 
		$\Delta \, T $ 	& Temperaturänderung 			& $[\Delta \, T ] = \kelvin$
\end{tabular}


\subsubsection{Volumenausdehnung $\bm{\Delta \, V}$}

\vspace{-0.2cm}

$$ \boxed{ V' = V + \Delta \,  V = V + \underbrace{  \gamma }_{\substack{\approx 3 \, \alpha}}  \cdot V \cdot \Delta \, T = V \, (1 + \gamma \cdot \Delta \, T ) }$$


\begin{tabular}{c l c}
		$V'$ 			& Volumen nach Ausdehnung 		& $[A'] = \meter^3$ 			\\
		$V$ 			& Anfangsvolumen 				& $[A] = \meter^3$ 				\\
		$\Delta \, V$ 	& Volumenänderung 				& $[\Delta \, V] = \meter^3$ 	\\
		$\gamma$ 		& Volumenausdehnungskoeffizient & $[\beta] = \frac{1}{\kelvin}$ \\ 
		$\Delta \, T$ 	& Temperaturänderung 			& $[\Delta \, T ] = \kelvin$
\end{tabular}


\subsection[Thermische Spannung]{Thermische Spannung $\bm{\sigma}$}

\vspace{-0.2cm}

$$\boxed{ p = \sigma = \varepsilon \cdot E = E \cdot \frac{\Delta l}{l} =  E \cdot \alpha \cdot \Delta \, T } $$

\renewcommand{\arraystretch}{1.3}
\begin{tabular}{c l c}
	$\sigma$ 		& Thermische Spannung 			& $[\sigma] = \pascal$ 				\\
	$\varepsilon$ 	& Dehnung 						& $[\varepsilon] = 1$ 				\\
	$E$ 			& Elastizitätsmodul 			& $[E] = \frac{\newton}{\meter^2}$ 	\\
	$\alpha$ 		& Längenausdehnungskoeffizient 	& $[\alpha] = \frac{1}{\kelvin}$ 	\\ 
	$\Delta \, T $ 	& Temperaturänderung 			& $[\Delta \, T ] = \kelvin$ 		\\
	$p$ 			& Druck 						& $[p] = \pascal$
\end{tabular}
\renewcommand{\arraystretch}{1}

