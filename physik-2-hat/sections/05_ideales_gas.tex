
\section{Ideales Gas}

\subsection{Modell des idealen Gases}

\textbf{Jedes Gas ist gleich!}

\medskip

\begin{enumerate}
	\item Moleküle sind Massepunkte (keine Ausdehnung)
	\item Stösse sind elastisch (keine zwischenmolekularen Kräfte) \\
		Kein Volumen bei $T = 0$ \\
		Kein Druck bei $T = 0$
\end{enumerate}


\subsubsection{Thermische Ausdehnung von Gasen}

\begin{itemize}
	\item Ausdehnung von Gasen ist sehr gross
	\item Bei \textbf{allen} Gasen ist die Ausdehnung \textbf{gleich}
	\item Volumen beim Nullpunkt ist \textbf{Null} 
\end{itemize}


\subsection{Universelle Gasgleichung}

Alle Gase verhalten sich gleich, insbesondere bei gleicher Anzahl Moleküle

$$ \boxed{ \frac{p \cdot V}{T} = \, \const \quad \Leftrightarrow \quad \frac{p_1 \cdot V_1}{T_1} = \frac{p_2 \cdot V_2}{T_2} } $$ 

\begin{tabular}{c l c}
	$p_x$ & \textbf{Absolut-}Druck $p_0 + p$ 	& $[p_x] = \pascal$ 	\\
	$V_x$ & Volumen 							& $[V_x] = \meter^3$ 	\\
	$T_x$ & \textbf{Absolut-}Temperatur			& $[T] = \kelvin$
\end{tabular}
	
	
\subsubsection{Boyle-Mariotte}	

\textbf{Das Gesetz gilt nur bei konstanter Temperatur!} \\
\textrightarrow\ \textbf{isotherme} Zustandsänderung

$$  \boxed{ p \cdot V = \, \const \quad \Leftrightarrow \quad p_1 \cdot V_1 = p_2 \cdot V_2 } $$ 


\subsubsection{Gay-Lussac}

\textbf{ Das Gesetz gilt nur bei konstantem Druck!} \\
\textrightarrow\ \textbf{isobare} Zustandsänderung

$$  \boxed{ \frac{V}{T} = \, \const \quad \Leftrightarrow \quad \frac{V_1}{T_1} = \frac{V_2}{T_2} } $$

	
\subsubsection{Gay-Lussac und Amontons}

\textbf{Das Gesetz gilt nur bei konstantem Volumen!} \\
\textrightarrow\  \textbf{isochore} Zustandsänderung

$$ \boxed{ \frac{p}{T} = \, \const \quad \Leftrightarrow \quad \frac{p_1}{T_1} = \frac{p_2}{T_2} } $$
	
	
\subsection{Universelle Gasgleichung für ideale Gase}

\vspace{-0.2cm}

$$ \boxed{ p \cdot V = n \cdot R \cdot T = N \cdot k \cdot T }$$
	
\renewcommand{\arraystretch}{1.3}
\begin{tabular}{c l c}
	$p$	& \textbf{Absolut-}Druck $p_0 + p$ 											& $[p] = \pascal$ 							\\
	$V$	& Volumen 																	& $[V] = \meter^3$ 							\\
	$n$	& Mol-Zahl 																	& $[n] = \mole$	 							\\
	$R$	& Universelle Gaskonstante $R = 8.314 \, \frac{\joule}{\mole \, \kelvin}$ 	& $[R] = \frac{\joule}{\mole \, \kelvin}$ 	\\
	$T$	& \textbf{Absolut-}Temperatur 												& $[T] = \kelvin$ 							\\
	$N$	& Anzahl Moleküle 															& $[N] = 1$ 								\\
	$k$	& Boltzmann-Konstante $k = 1.381 \cdot 10^{-23} \, \frac{\joule}{\kelvin}$ 	& $[k] = \frac{\joule}{\kelvin}$
\end{tabular}
\renewcommand{\arraystretch}{1}
	
	
\subsubsection{Zusammenhänge zwischen den Konstanten}
	
\vspace{-0.2cm}

$$ \boxed{ R = k \cdot N_A = \frac{N \cdot k}{n} } \qquad \qquad \boxed{ n = \frac{N}{N_A} = \frac{m}{M} = \frac{N \cdot k}{R} }$$
	

\renewcommand{\arraystretch}{1.3}
\begin{tabular}{c l c}
	$R$		& Universelle Gaskonstante $R = 8.314 \, \frac{\joule}{\mole \, \kelvin}$ 	& $[R] = \frac{\joule}{\mole \, \kelvin}$ 	\\
	$k$		& Boltzmann-Konstante $k = 1.381 \cdot 10^{-23} \, \frac{\joule}{\kelvin}$ 	& $[k] = \frac{\joule}{\kelvin}$			\\
	$N$ 	& Anzahl Moleküle 															& $[N] = 1$ 								\\
	$N_A$ 	& Avogadrokonstante: $N_A = 6.022 \cdot 10^{23} \, \frac{1}{\mole}$ 		& $[N_A] = \frac{1}{\mole}$					\\	
	$n$ 	& Mol-Zahl 																	& $[n] = \mole$ 							\\
	$m$ 	& Masse 																	& $[m] = \kilogram$ 						\\
	$M$ 	& Mol-Masse 																& $[M] = \frac{\kilogram}{\mole}$
\end{tabular}
\renewcommand{\arraystretch}{1}



\subsection[Mechanische Arbeit von Gasen]{Mechanische Arbeit $\bm{\Delta W}$ von Gasen}
\label{MechArbeit}

Folgende Formel ist für Flüssigkeiten \textbf{nicht} gültig, da diese inkompressibel sind ($\Delta V = 0$)

$$ \boxed{ \Delta W = F \cdot \Delta s = p \cdot A \cdot \Delta s = p \cdot \Delta V } $$


\begin{tabular}{c l c}
	$\Delta W$ 	& Mechanische Arbeit von Gas	& $[\Delta W] = \joule$		\\
	$F$ 		& Kraft 						& $[F] = \newton$ 			\\
	$\Delta s$ 	& Wegänderung 					& $[\Delta s] = \meter$ 	\\
	$p$ 		& Druck 						& $[p] = \pascal$ 			\\
	$A$ 		& Fläche 						& $[A] = \meter^2$ 			\\
	$\Delta V$ 	& Volumenänderung 				& $[\Delta V] = \meter^3$
\end{tabular}


\subsection{Gesetz von Avogadro}

Ein Mol eines Gases nimmt bei Normalbedingungen immer das gleiche Volumen ein (=Molvolumen) 

\medskip

Ideale Gase enthalten bei gleichem Druck $p$ und gleicher Temperatur $T$ immer gleich viele Moleküle (im Molvolumen)



\subsection[Molmasse / Molvolumen]{Molmasse $\bm{M}$, Molvolumen $\bm{V_m}$}

Für 1 Mol Teilchen gilt: 

$$ \boxed{ p \cdot V = R \cdot T = N_A \cdot k \cdot T } $$

\medskip

\begin{minipage}[t]{0.58\columnwidth}
	\subsubsection{Molmasse}

	Entspricht der \textbf{Ordnungszahl} im Periodensystem
	$$  \boxed{ n = \frac{m}{M} = \frac{N}{N_A} } $$
\end{minipage}
\hfill
\begin{minipage}[t]{0.38\columnwidth}
	\subsubsection{Molvolumen}
	
	\phantom{eifach öppis}
	$$ \boxed{ V_m = \frac{V}{n} }$$
\end{minipage}


\renewcommand{\arraystretch}{1.3}
\begin{tabular}{c l c}
	$p$ 	& \textbf{Absolut-}Druck $p_0 + p$ 											& $[p] = \pascal$ 							\\
	$V$ 	& Volumen 																	& $[V] = \meter^3$ 							\\
	$R$		& Universelle Gaskonstante $R = 8.314 \, \frac{\joule}{\mole \, \kelvin}$ 	& $[R] = \frac{\joule}{\mole \, \kelvin}$ 	\\
	$T$ 	& \textbf{Absolut-}Temperatur 												& $[T] = \kelvin$ 							\\
	$N_A$	& Avogadrokonstante: $N_A = 6.022 \cdot 10^{23} \, \frac{1}{\mole}$ 		& $[N_A] = \frac{1}{\mole}$					\\	
	$k$		& Boltzmann-Konstante $k = 1.381 \cdot 10^{-23} \, \frac{\joule}{\kelvin}$ 	& $[k] = \frac{\joule}{\kelvin}$			\\
	$n$ 	& Mol-Zahl 																	& $[n] = \mole$ 							\\
	$m$ 	& Masse 																	& $[m] = \kilogram$ 						\\
	$M$ 	& Mol-Masse 																& $[M] = \frac{\kilogram}{\mole}$			\\
	$N$ 	& Anzahl Moleküle 															& $[N] = 1$ 								\\
	$V_m$ 	& Mol-Volumen 																& $[V_m] = \frac{\meter^3}{\mole}$
\end{tabular}
\renewcommand{\arraystretch}{1}


\subsection[Dichte eines Gases]{Dichte eines Gases $\bm{\rho}$}

\vspace{-0.2cm}

$$ \boxed{ \rho = \frac{m}{V} = \frac{M}{V_m} = \frac{p \cdot M}{R \cdot T} }$$

\renewcommand{\arraystretch}{1.3}
\begin{tabular}{c l c}
	$\rho$ 	& Gas-Dichte 																		& $[\rho] = \frac{\kilogram}{\meter^3}$		\\
	$m$ 	& Masse 																			& $[m] = \kilogram$ 						\\
	$V$ 	& Volumen 																			& $[V] = \meter^3$ 							\\
	$M$ 	& Mol-Masse 																		& $[M] = \frac{\kilogram}{\mole}$			\\
	$V_m$ 	& Mol-Volumen ($22.4 \, \liter$ bei $0 \, \celsius$ und $1000 \, \hecto \pascal)$	& $[V_m] = \frac{\meter^3}{\mole}$ 			\\
	$p$ 	& \textbf{Absolut-}Druck $p_0 + p$ 													& $[p] = \pascal$ 							\\
	$R$		& Universelle Gaskonstante $R = 8.314 \, \frac{\joule}{\mole \, \kelvin}$ 			& $[R] = \frac{\joule}{\mole \, \kelvin}$ 	\\
	$T$ 	& \textbf{Absolut-}Temperatur 														& $[T] = \kelvin$
\end{tabular}
\renewcommand{\arraystretch}{1}


\subsection{Phänomene von idealen Gasen}

\subsubsection{Annomalie des Wassers}

Die feste Form (Eis) ist leichter als die flüssige Form (Wasser) \\
Die \textbf{grösste Dichte weist Wasser bei 4 °C} auf, nicht beim Gefrierpunkt von $0 \, \celsius$

\smallskip

\textrightarrow\ Ein See gefriert somit nur an der Oberfläche. Am Grund des Sees beträgt die Wassertemperatur $4 \, \celsius$ 


\subsubsection{Osmotischer Druck (Zelldruck)}

Grosse Moleküle innerhalb von vielen kleinen Molekülen in einer Flüssigkeit verhalten sich ähnlich wie die Moleküle eines idealen Gases, 
wenn die Flüssigkeit von einer für die Müleküle halb-durchlässigen (semi-permeabel) Membran umgeben ist.

$$ \text{Osmotischer Druck:} \quad p = \frac{n}{V} \cdot R \cdot T  \qquad \text{(ideale Gasgleichung)}$$


\subsection[Partialdruck]{Partialdruck $\bm{p_i}$}

\textbf{Ausgangslage: Gasgemisch (z.B. Luft: Sauerstoff-Stickstoff)}

\begin{minipage}[c]{0.48\columnwidth}
	\includegraphics[width=\columnwidth,]{images/partialdruck.png}
\end{minipage}
\hfill
\begin{minipage}[c]{0.48\columnwidth}
	Der Partialdruck $p_i$ ist der Druck, welcher die $i$-te Gaskomponete erzeugen würde, wenn ihr das gesamte Volumen zur 
	Verfügung stehen würde.
\end{minipage}



\subsection{Gesetz von Dalton}

In einem Gas ist die Summe der Partialdrücke $p_i$ gleich dem Gesamtdruck 

\begin{minipage}[c]{0.38\columnwidth}
	$$ \boxed{ \sum_{i=1}^n  p_i = p } $$
\end{minipage}
\hfill
\begin{minipage}[c]{0.58\columnwidth}
	\begin{tabular}{c l c}
		$p_i$	& Partialdruck 		& $[p_i] = \pascal$	\\
		$p$ 	& (Gesamt-) Druck 	& $[p] = \pascal$ 	
	\end{tabular}
\end{minipage}
 


\subsection{Volumen- und Massenkonzentration (Gasgemisch)}

\subsubsection{Volumen-Konzentrationen (Volumen-Anteile)}

\vspace{-0.2cm}

$$ \boxed{  q_i = \frac{V_i}{V} = \frac{n_i}{n} = \frac{p_i}{p} } $$


\begin{tabular}{c l c}
	$q_i$ 	& Volumen-Konzentration						& $[q_i] = 1$ 			\\
	$V_i$ 	& Volumen der $i$-ten Gas-Komponente 		& $[V_i] = \meter^3$ 	\\
	$V$ 	& Gesamt-Volumen	 						& $[V] = \meter^3$ 		\\
	$n_i$ 	& Molzahl der $i$-ten Gas-Komponente 		& $[n_i] = \mole$ 		\\
	$n$ 	& Gesamt-Molzahl des Gemischs 				& $[n] = \mole$ 		\\
	$p_i$ 	& Partialdruck der $i$-ten Gaskomponente 	& $[p_i] = \pascal$ 	\\
	$p$ 	& Druck des Gemischs 						& $[p] = \pascal$
\end{tabular}


\subsubsection{Massen-Konzentration (Massen-Anteile)}

\vspace{-0.2cm}

$$ \boxed{ \mu_i = \frac{m_i}{m} = \frac{M_i}{M} \cdot q_i } $$

\renewcommand{\arraystretch}{1.3}
\begin{tabular}{c l c}
	$\mu_i$ & Volumen-Konzentrationen 				& $[\mu_i] = 1$ 					\\
	$m_i$ 	& Masse der $i$-ten Gas-Komponente 		& $[m_i] = \kilogram$ 				\\
	$m$ 	& Masse der Gemischs 					& $[m] = \kilogram$ 				\\
	$M_i$ 	& Mol-Masse der $i$-ten Gas-Komponete 	& $[M_i] = \frac{\kilogram}{\mole}$ \\
	$M$ 	& Mol-Masse des Gemischs 				& $[M] = \frac{\kilogram}{\mole}$ 	\\
	$q_i$ 	& Volumen-Konzentration					& $[q_i] = 1$
\end{tabular}
\renewcommand{\arraystretch}{1}


\subsection{Mol-Masse Gasgemisch}

Die Mol-Masse des Gas-Gemischs kann als gewichteter Mittelwert berechnet werden, gewichtet mit den jeweiligen Volumen-Anteilen.

$$ \boxed{ M = \sum_{i=1}^n  q_i \cdot M_i } $$

\renewcommand{\arraystretch}{1.3}
\begin{tabular}{c l c}
	$M$ 	& Mol-Masse Gasgemisch								& $[M] = \frac{\kilogram}{\mole}$ 	\\
	$q_i$ 	& Volumen-Konzentration der $i$-ten Gas-Komponente	& $[q_i] = 1$ 						\\
	$M_i$ 	& Mol-Masse der $i$-ten Gas-Komponete 				& $[M_i] = \frac{\kilogram}{\mole}$
\end{tabular}
\renewcommand{\arraystretch}{1}

