\section{Reales Gas}

Im vergleich zum idealen Gas müssen zwei Dinge berücksichtigt werden:

\medskip

\begin{description}
	\item[Eigen-Volumen:] Ideales Gas hat \textbf{kleineres} Volumen als gemessen \\
		( \textrightarrow\ Ideal-Gas-Volumen um das Molekül-Eigenvolumen reduzieren)
	\item[Binnen-Druck:] Ideales Gas hat \textbf{grösseren} Druck als gemessen \\
		( \textrightarrow\ Ideal-Gas-Druck um Binnendruck erhöhen) 
\end{description}


\subsection{Van der Waals-Gleichung (1 mol)}

\textbf{\textrightarrow\ Für nicht-ideale Gase!} 

$$ \boxed{ p' \cdot V'_m = R \cdot T } \qquad \qquad \boxed{ p' = p + \frac{a}{V_m^2} } \qquad \qquad \boxed{ V'_m = V_m - b } $$


\renewcommand{\arraystretch}{1.3}
\begin{tabular}{c l c}
	$p'$ 	& Korrigierter Druck 														& $[p'] = \pascal$ 								\\
	$V'_m$ 	& Korrigiertes Mol-Volumen 													& $[V_m] = \frac{\meter^3}{\mole}$ 				\\
	$R$		& Universelle Gaskonstante $R = 8.314 \, \frac{\joule}{\mole \, \kelvin}$ 	& $[R] = \frac{\joule}{\mole \, \kelvin}$ 		\\
	$T$ 	& \textbf{Absolut-}Temperatur		 										& $[T] = \kelvin$ 								\\
	$p$ 	& Druck des Gemischs 														& $[p] = \pascal$ 								\\
	$a$ 	& Eigenvolumen 																& $[a] = \frac{\joule \, \meter^3}{\mole^2}$ 	\\
	$b$ 	& Binnendruck 																& $[b] = \frac{\meter^3}{\mole}$ 				\\
	$V_m$ 	& Mol-Volumen 																& $[V_m] = \frac{\meter^3}{\mole}$
\end{tabular}
\renewcommand{\arraystretch}{1}


\subsection[Van der Waals-Gleichung (n Mol)]{Van der Waals-Gleichung ($\bm{n}$ mol)}

\vspace{-0.2cm}

$$ \boxed{ \Big( p + \frac{n^2 \cdot a}{V^2} \Big)  \cdot (V - n \cdot b) = n \cdot R \cdot T } $$

\renewcommand{\arraystretch}{1.3}
\begin{tabular}{c l c}
	$p$ & Druck des Gemischs 															& $[p] = \pascal$ 								\\
	$n$ & Mol-Zahl 																		& $[n] = \mole$ 								\\
	$a$ & Eigenvolumen 																	& $[a] = \frac{\joule \, \meter^3}{\mole^2}$ 	\\
	$V$ & Volumen 																		& $[V] = \meter^3$ 								\\
	$b$ & Binnendruck 																	& $[b] = \frac{\meter^3}{\mole}$ 				\\
	$R$	& Universelle Gaskonstante $R = 8.314 \, \frac{\joule}{\mole \, \kelvin}$ 		& $[R] = \frac{\joule}{\mole \, \kelvin}$ 		\\
	$T$ & \textbf{Absolut-}Temperatur 													& $[T] = \kelvin$
\end{tabular}
\renewcommand{\arraystretch}{1}


\subsubsection{Van der Waals-Parameter}

\vspace{-0.2cm}
$$ \boxed{ a = \frac{9}{8} \cdot R \cdot T_k \cdot V_{mk} }  \qquad \qquad  \boxed{ b = \frac{V_{mk}}{3} }$$
$$ \boxed{ V_{mk} = 3 \cdot b } \quad \quad \boxed{ T_k = \frac{8 \cdot a}{27 \cdot R \cdot b} } \quad \quad  \boxed{ p_k = \frac{a}{27 \cdot b^2} } $$


\renewcommand{\arraystretch}{1.3}
\begin{tabular}{c l c}
	$a$ 		& Eigenvolumen 																	& $[a] = \frac{\joule \, \meter^3}{\mole^2}$ 	\\
	$R$			& Universelle Gaskonstante $R = 8.314 \, \frac{\joule}{\mole \, \kelvin}$ 		& $[R] = \frac{\joule}{\mole \, \kelvin}$ 		\\
	$T_k$ 		& Kritische \textbf{Absolut-} Temperatur 										& $[T_k] = \kelvin$ 							\\
	$V_{mk}$ 	& Kritisches Mol-Volumen 														& $[V_{mk}] = \frac{\meter^3}{\mole}$ 			\\
	$b$ 		& Binnendruck 																	& $[b] = \frac{\meter^3}{\mole}$ 				\\
	$p_k$ 		& Kritischer Druck 																& $[p_k] = \pascal$
\end{tabular}
\renewcommand{\arraystretch}{1}

