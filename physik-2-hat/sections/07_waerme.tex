\section{Wärmelehre}

\subsection{Wärme Q}

Wärme ist Energie, welche stets \textbf{(von allein)} von höherer zu niederigerer Temperatur fliesst.



\begin{ctabular}{l c l}
 				& $\underleftarrow{1. \text{ HS } 100\%}$ 		& 							\\
	$\Delta U$ 	& $=$ 											& $\Delta W + \Delta Q$ 	\\
				& $\overrightarrow{2 \text{ HS } \xout{100\% }}$
\end{ctabular}


\subsection{Erster Hauptsatz der Wärmelehre}

Nicht nur durch Wärmezufuhr, sondern auch durch mechanische Arbeit lässt sich die Temperatur und damit die innere Energie $U$ erhöhen.

$$ \boxed{ \Delta U =\Delta W + \Delta Q } $$

\begin{tabular}{c l c}
	$\Delta U$ 	& Zu-/Abgeführte Innere Energie 															& $[\Delta U] = \joule$ \\
	$\Delta W$ 	& Zu-/Abgeführte Arbeit z.B. $E_{\rm kin}, \; E_{\rm pot}, \; W_{\rm Gas}, \; W_{\rm reib}$	& $[\Delta W] = \joule$ \\
	$\Delta Q$ 	& Zu-/Abgeführte Wärme 																		& $[\Delta Q] = \joule$
\end{tabular}


\subsubsection{Ansätze für 1. HS}

\vspace{-0.2cm}

$$ \Delta Q = E_{\rm kin} = \frac{1}{2} \, m \cdot v^2 \qquad \qquad \Delta Q = E_{\rm pot} = m \cdot g \cdot h \qquad \qquad \Delta \dot{Q} = \Delta P $$


\subsubsection{Mechanische Arbeit eines Gases}

Für mehr Details, siehe Abschnitt ~\ref{MechArbeit}

$$  \boxed{\Delta W = p \cdot \Delta V } $$


\subsection{Mechanische Wärmeäquivalente}

1 Kalorie = $4,1868 \, \joule$ (cal) \\
\textrightarrow\ Energie, um 1 Gramm Wasser um 1 Grad zu erwärmen

\medskip

1 kcal = $4186,8 \, \joule$ \\
\textrightarrow\ Energie, um 1 Kilogramm Wasser um 1 Grad zu erwärmen


\subsubsection{Elektrisches Wärmeäquivalent $\bm{c}$}

\textbf{Elektrische Energie = Wärme}

$$ \boxed{ U \cdot I \cdot t = c \cdot m \cdot \Delta T \quad \Leftrightarrow \quad c = \frac{U \cdot I \cdot t}{m \cdot \Delta T} } $$


\begin{tabular}{c l c}
	$c$ 		& Elektrisches Wärmeäquivalent	& $[c] = \frac{\joule}{\kilogram \, \kelvin}$		\\
	$U$ 		& Spannung			 			& $[U] = \volt$ 									\\
	$I$ 		& Strom 						& $[I] = \ampere$ 									\\
	$t$ 		& Zeit 							& $[t] = \second$			 						\\
	$m$ 		& Masse 						& $[m] = \kilogram$ 								\\
	$\Delta T$ 	& Temperaturänderung 			& $[\Delta T] = \kelvin$
\end{tabular}


\subsection{Wärmekapazität}

Die Wärmekapazität drückt das Energiespeicher-Vermögen aus.

$$ \boxed{ Q = c \cdot m \cdot \Delta T = n \cdot C_m \cdot \Delta T = C \cdot \Delta T } $$


\subsubsection{Absolute Wärmekapazität $\bm{C}$}

Energiespeicher-Vermögen eines \textbf{Gegenstands}

$$ \boxed{ \Delta Q = C \cdot \Delta T } $$


\subsubsection{Spezifische Wärmekapazität $\bm{c}$}

Energiespeicher-Vermögen einer \textbf{Substanz}

$$ \boxed{ \Delta Q = c \cdot m \cdot \Delta T } \qquad \qquad c_{\rm Wasser} = 4187 \frac{\joule}{\kilogram \, \kelvin} $$


\subsubsection{Molare Wärmekapazität $\bm {C_m}$}

Energiespeicher-Vermögen einer \textbf{Anzahl Moleküle}

$$ \boxed{ C_m = \frac{c}{n} = M \cdot c } $$

\renewcommand{\arraystretch}{1.3}
\begin{tabular}{c l c}
	$\Delta Q$	& Zu-/Abgeführte Wärme 			& $[\Delta Q] = \joule$ 						\\
	$c$ 		& Spezifische Wärmekapazität 	& $[c] = \frac{\joule}{\kilogram \, \kelvin}$ 	\\
	$C$ 		& Absolute Wärmekapazität 		& $[C] = \frac{\joule}{\kelvin}$ 				\\
	$C_m$ 		& Molare Wärmekapazität 		& $[C_m] = \frac{\joule}{\mole \, \kelvin}$ 	\\
	$m$ 		& Masse 						& $[m] = \kilogram$ 							\\
	$\Delta T$ 	& Temperaturänderung 			& $[\Delta T] = \kelvin$ 						\\
	$n$ 		& Mol-Zahl 						& $[n] = \mole$ 								\\
	$M$ 		& Mol-Masse 					& $[M] = \frac{\kilogram}{\mole}$
\end{tabular}
\renewcommand{\arraystretch}{1}


\subsubsection{Molare Wärmekapazität von Gasen}

\vspace{-0.2cm}

$$ \boxed{ C_{mp} - C_{mV} = R } $$

\renewcommand{\arraystretch}{1.3}
\begin{tabular}{c l c}
	$C_{mp}$ 	& Isobare Wärme-Kapazität $(p = \const)$ 									& $[C_{mp}] = \frac{\joule}{\mole \, \kelvin}$	\\
	$C_{mV}$ 	& Isochore Wärme-Kapazität $(V = \const)$ 									& $[C_{mV}] = \frac{\joule}{\mole \, \kelvin}$	\\
	$R$			& Universelle Gaskonstante $R = 8.314 \, \frac{\joule}{\mole \, \kelvin}$ 	& $[R] = \frac{\joule}{\mole \, \kelvin}$
\end{tabular}
\renewcommand{\arraystretch}{1}


\subsubsection{Molare Wärmekapazität von Festkörpern}

\vspace{-0.4cm}

\begin{align*}
	T > \Theta_D: 	\quad &C_m \approx 3 \, R \approx 25 \, \frac{\joule}{\mole}  &\text{ (Dulung-Petit)} \\
	T \ll \Theta_D:	\quad &C_m = \frac{12 \cdot \pi^4}{5}  \cdot R \cdot  \Bigg( \frac{T}{\Theta_D}  \Bigg)^3  &\text{ (Debye)} 
\end{align*}


\renewcommand{\arraystretch}{1.3}
\begin{tabular}{c l c}
	$T$ 		& \textbf{Absolut-}Temperatur 												& $[T] = \kelvin$ 							\\
	$\Theta_D$ 	& Debye-Temperatur $\Theta_D \approx 200 \, \kelvin$ 						& $[\Theta_D] = \kelvin$ 					\\
	$C_m$ 		& Molare Wärmekapazität 													& $[C_m] = \frac{\joule}{\mole \, \kelvin}$ \\
	$R$			& Universelle Gaskonstante $R = 8.314 \, \frac{\joule}{\mole \, \kelvin}$ 	& $[R] = \frac{\joule}{\mole \, \kelvin}$
\end{tabular}
\renewcommand{\arraystretch}{1}


\subsection{Latente Wärme (Schmelz-/ Verdampfungswärme)}

\begin{minipage}[t]{0.48\columnwidth}
	\includegraphics[width=\columnwidth, align=t]{images/latente_waerme_2.png}
\end{minipage}
\hfill
\begin{minipage}[t]{0.48\columnwidth}
	Beim Schmelzen und Verdampfen findet \textbf{keine} Temperaturerhöhung statt.

	\medskip

	Beim Gefrieren und oder Kondensieren wird diese versteckte Wärme wieder frei, \textbf{ohne} Abnahme der Temperatur
\end{minipage}

\medskip

\textbf{\textrightarrow\ Die Schmelz-/ Verdampfungswärme ist stark druckabhängig}


$$ \boxed{ Q_f = q_f \cdot m } \qquad \qquad q_{f_{\rm Wasser}} := 334  \, \frac{\kilo \joule}{\kilogram} $$
$$ \boxed{ Q_S = q_s \cdot m } \qquad \qquad q_{s_{\rm Wasser}} := 2256 \, \frac{\kilo \joule}{\kilogram} $$


\renewcommand{\arraystretch}{1.3}
\begin{tabular}{c l c}
	$Q_f$ 	& Schmelz-/Erstarrungs-Wärme	 	& $[Q_f] = \joule$ 						\\
	$q_f$ 	& Spezifische Schmelzwärme 			& $[q_f] = \frac{\joule}{\kilogram}$	\\
	$Q_S$ 	& Verdampfungs-/Kondensations-Wärme & $[Q_S] = \joule$ 						\\
	$q_s$ 	& Spezifische Verdampfungs-Wärme	& $[q_s] = \frac{\joule}{\kilogram}$	\\
	$m$ 	& Masse 							& $[m] = \kilogram$
\end{tabular}
\renewcommand{\arraystretch}{1}



\subsection{Wärmebilanz}

Wärmeaustausch zwischen verschiedenen Materialien

\smallskip

In einem abgeschlossenen System (nach aussen isoliert) muss gelten: \\
\textbf{Zugeführte Wärme = Abgeführte Wärme}

$$  \boxed{ \sum_{i=1}^n  ( \Delta Q_i + \Delta Q_{f_i} + \Delta Q_{s_i} ) = 0 } $$


\begin{tabular}{c l c@{}}
	$\Delta Q_i$		& $i$-te Wärme-Menge aus Temperatur-Zu-/Abnahme					& $[\Delta Q_i] = \joule$ 			\\
	$\Delta Q_{f_i}$ 	& $i$-te Wärme-Menge aus Schmelz-/Erstarrungs-Vorgang			& $[\Delta Q_{f_i}] = \joule$ 		\\
	$\Delta Q_{s_i}$ 	& $i$-te Wärme-Menge aus Verdampfungs-/Kondensations-Vorgang 	& $[\Delta Q_{s_i}] = \joule$		\\
	$+$					& Zugeführte Wärme-Menge  																			\\
	$-$					& Abgeführterr Wärme-Menge
\end{tabular}

