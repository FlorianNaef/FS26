\section{Rückwandlung innerer Energie}

\subsection{Zweiter Hauptsatz der Wärmelehre}

\begin{itemize}
	\item Innere Energie kann \textbf{nicht zu 100 \%} in Arbeit umgesetzt werden \\
		\textrightarrow\ Carnot-Wirkungsgrad ist der theoretisch höchstmögliche
	\item Wärme kann niemals \myul{von selbst} von einem kälteren Ort zu einem wärmeren Ort fliessen (Clausius)
	\item Es gibt keine \myul{periodisch wirkende} Maschine, die nichts anderes bewirkt als Erzeugung mechanischer Arbeit und 
		Abkühlung eines Wärme-Reservoirs (Kelvin) \\
		\textrightarrow\ Es gibt kein Perpetuum mobile 2. Art
\end{itemize}


\subsection{Kreisprozess (reversibler Prozess)}

\begin{center}
	Anfangszustand = Endzustand
\end{center}


\begin{minipage}[t]{0.48\columnwidth}
	\myul{\textbf{Rechts}laufender Kreisprozess}

	\smallskip

	\begin{itemize}
		\item Gibt Arbeit ab
		\item \cbl{Wärmekraftmaschine}
		\item Bei hoher Temperatur wird Wärme aus Prozess \textbf{zu}geführt
		\item Nur Bruchteil der Wärme in Arbeit verwandelbar
		\item Obergrenze: Carnot-Wirkungsgrad
	\end{itemize}
\end{minipage}
\hfill
\begin{minipage}[t]{0.48\columnwidth}
	\myul{\textbf{Links}laufender Kreisprozess}

	\smallskip

	\begin{itemize}
		\item Verbraucht Arbeit
		\item \cvt{Wärmepumpe}
		\item Bei hoher Temperatur wird dem Prozess Wärme \textbf{ab}geführt
		\item Erzeugt mehrfaches an Wärme
		\item Obergrenze: Inv. Carnot-Wirkungsgrad
	\end{itemize}
\end{minipage}





\subsection{Carnot-Wirkungsgrad}

\vspace{-0.2cm}

$$ \boxed{\text{\cbl{Wärmekraftmaschine:}}  \quad n_C = \frac{W_{\rm ab}}{Q_{\rm zu}} = \frac{T_{\rm hoch} - T_{\rm tief}}{T_{\rm hoch}} } $$

$$ \boxed{\text{ \cvt{Wärmepumpe:}} \quad  n_{iC} = \frac{Q_{\rm zu}}{W_{\rm ab}} = \frac{T_{\rm hoch}}{T_{\rm hoch} - T_{\rm tief}} } $$


\begin{tabular}{c l c}
	$n_C$ 			& Carnot-Wirkungsgrad 				& $[n_C] = 1$ 					\\
	$n_{iC}$ 		& Inverset Carnot-Wirkungsgrad 		& $[n_{iC}] = 1$ 				\\
	$T_{\rm tief}$ 	& Temperatur des Warm-Reservoirs 	& $[T_{\rm tief}] = \kelvin$	\\
	$T_{\rm hoch}$ 	& Temperatur des Kalt-Reservoirs 	& $[T_{\rm hoch}] = \kelvin$ 	\\
	$Q_{\rm zu}$ 	& zugeführte Wärme 					& $[Q_{\rm zu}] = \joule$ 		\\ 
	$W_{\rm ab}$ 	& abgeführte Energie 				& $[W_{\rm ab}] = \joule$
\end{tabular}


\subsection{Adiabaten-Gleichung (Kreisprozess)}

Adiabate wird beschrieben im pV- / TV- / Tp-Diagramm

\medskip

\begin{minipage}{0.6\columnwidth}
	\includegraphics[width=\columnwidth]{images/kreisprozess.png}
\end{minipage}
\hfill
\begin{minipage}{0.38\columnwidth}
	$$ \boxed{ p \cdot V^\kappa  = \const } $$
	$$ \boxed{ T \cdot V^{\kappa -1} = \const } $$
	$$ \boxed{ T^\kappa \cdot p^{1-\kappa} = \const } $$
	$$ \kappa = \frac{C_{mp}}{C_{mV}} $$
	$$ \boxed{ C_{mp} - C_{mV} = R } $$
\end{minipage}

\medskip

\renewcommand{\arraystretch}{1.3}
\begin{tabular}{c l c}
	$C_{mp}$ 	& Molare Wärmekapazität @ $p = \const$ 										& $[C_{mp}] = \frac{\joule}{\mole \, \kelvin}$	\\
	$C_{mV}$ 	& Molare Wärmekapazität @ $V = \const$  									& $[C_{mV}] = \frac{\joule}{\mole \, \kelvin}$	\\
	$\kappa$ 	& Adiabaten-Exponent 														& $[\kappa] = 1$ 								\\
	$R$			& Universelle Gaskonstante $R = 8.314 \, \frac{\joule}{\mole \, \kelvin}$ 	& $[R] = \frac{\joule}{\mole \, \kelvin}$
\end{tabular}
\renewcommand{\arraystretch}{1}


\subsection{Kreisprozesse (Vorgänge)}


\begin{tabular}{lll}
	isotherme Expansion 		& liefert Wärme 	& benötigt Energie	\\
	isotherme Kompression 		& benötigt Wärme 	& liefert Energie 	\\
	\\
	adiabatische Expansion 		& liefert Arbeit 	& ohne Wärme 		\\
	adiabatische Kompression 	& benötigt Arbeit 	& ohne Wärme 		\\
	\\
	isochore Erwärmung 			& ohne Arbeit 		& benötigt Wärme 	\\
	isochore Abkühlung 			& ohne Arbeit 		& liefert Wärme
\end{tabular}


\subsection{Beispiel Kreisprozess}

\begin{minipage}[c]{0.48\columnwidth}
	\includegraphics[width=\columnwidth]{images/kreisprozess_2.png}
\end{minipage}
\hfill
\begin{minipage}[c]{0.48\columnwidth}
	\includegraphics[width=\columnwidth]{images/kreisprozess_3.png}
\end{minipage}


\subsection{Entropie-Zunahme}

\subsubsection{Definition der Entropie-Zunahme}

\vspace{-0.2cm}

$$ \Delta S = S_1 + S_2 = \int \frac{1}{T} \, \diff Q $$


\subsubsection{Boltzmann-Gleichung für Entropie-Zunahme}

\vspace{-0.2cm}

$$ \boxed{ \Delta S = k \cdot \ln(W) } $$


\renewcommand{\arraystretch}{1.3}
\begin{tabular}{c l c}
	$\Delta S$	& Entropie 																	& $[\Delta S] = \frac{\joule}{\kelvin}$	\\
	$k$ 		& Boltzmann-Konstante $k = 1.381 \cdot 10^{-23} \, \frac{\joule}{\kelvin}$	& $[k] = \frac{\joule}{\kelvin}$		\\
	$W$ 		& Wahrscheinlichkeit eines Zustands 										& $[W] = 1$
\end{tabular}
\renewcommand{\arraystretch}{1}


\subsubsection{Abgeschlossenes System}

\renewcommand{\arraystretch}{1.2}
\begin{tabular}{ll}
	$ \Delta S \geq 0$	& Entropie kann nur zunehmen in abgeschlossenem System	\\
	$ \Delta S > 0$ 	& Irreversibler Prozess 								\\
	$ \Delta S = 0$ 	& Reversibler Prozess
\end{tabular}
\renewcommand{\arraystretch}{1}

\columnbreak

