\section{Molmassen wichtiger Atome}

\renewcommand{\arraystretch}{1.3}
\begin{tabular}{c | c | c }
    \textbf{Symbol} & \textbf{Molekül}  & \textbf{Molmasse}                 \\
    \hline
    H               & Wasserstoff       & $1.008 \, \frac{\gram}{\mole}$    \\
    C               & Kohlenstoff       & $12.011 \, \frac{\gram}{\mole}$   \\
    N               & Stickstoff        & $14.007 \, \frac{\gram}{\mole}$   \\
    O               & Sauerstoff        & $15.999 \, \frac{\gram}{\mole}$   \\
    Al              & Aluminium         & $26.982 \, \frac{\gram}{\mole}$   \\
    Si              & Silicium          & $28.982 \, \frac{\gram}{\mole}$   \\
\end{tabular}
\renewcommand{\arraystretch}{1.3}


\section{Ansätze zu Aufgaben}

\subsection{Barometer}

\begin{minipage}[t]{0.4\columnwidth}
    \includegraphics[width=\columnwidth, align=t]{images/manometer.jpg}
\end{minipage}
\hfill
\begin{minipage}[t]{0.5\columnwidth}
    $$ \boxed{ p_1 = p_0 + \underbrace{ \rho_{Fl} \cdot g \cdot h}_{\substack{p_s}} } $$
 
    \begin{tabular}{ll}
        $p_1$   & Gemessener Druck  \\
        $p_0$   & Luftdruck         \\
        $p_s$   & Schweredruck
    \end{tabular}

    \medskip

    \textrightarrow\ Bernoulli \\
    \textrightarrow\ Kontinuität
\end{minipage}




\subsection{Pitotrohr}
Prandtl'sches Staurohr; Staudruckmesser \\
Zur Messung von Strümungsgeschwindigkeiten 

\begin{center}
    \includegraphics[width=0.73\columnwidth]{images/pitotrohr.png}
\end{center}


$$ \boxed{ \text{Bernoulli horizontal:} \quad \underbrace{p_1}_{\substack{p_L}} + \frac{1}{2} \, \rho_1 \cdot \underbrace{v_1^2}_{\substack{0}} = \underbrace{p_2}_{\substack{p_L - \Delta p}} + \frac{1}{2} \, \underbrace{\rho_2}_{\substack{\rho_L}} \cdot v_2^2} $$

$$ 0 = - \Delta p + \frac{1}{2} \, \rho_L \cdot v_2^2 \qquad \Rightarrow \Delta p =\frac{1}{2} \, \rho_L \cdot v_2^2 $$

$$ \text{Gleichsetzen:} \quad \Delta p = \rho_{Fl} \cdot g \cdot h $$


\subsection{Pumpe}

\vspace{-0.2cm}

$$ \boxed{ W = P \cdot t = F \cdot \Delta s = p \cdot A \cdot \Delta s = p \cdot \Delta V } \qquad  \qquad \boxed{ F = p \cdot A }  $$

$$ \boxed{ P =  \frac{W}{t} = \frac{p \cdot V}{t} = p  \cdot  \dot{V} } $$


\subsection{Bewegungen}

\vspace{-0.2cm}

$$ \boxed{ P = F \cdot v } \qquad \qquad \boxed{ E_{\rm kin} = \frac{1}{2} \, m \cdot v^2 } $$



\subsection{U-Rohr}

\begin{minipage}[t]{0.35\columnwidth}
\includegraphics[width=\columnwidth, align=t]{images/u-rohr.png}
\end{minipage}
\hfill
\begin{minipage}[t]{0.6\columnwidth}

    Ansatz: Druckgleichgewicht 

    $$ \boxed{ p_1 = p_2 } $$

    $$ \boxed{ \rho_1 \cdot g \cdot h_1 = \rho_2 \cdot g \cdot h_2 } $$
\end{minipage}


\subsection{Wasser mit Dampf erhitzen}

Ein Tasse mit $m_W = 200 \, \gram$ Wasser  mit einer Temperatur von $T_K = 20 \, \celsius$ wird an
der Wasserdampfdüse einer Kaffeemaschine mittels Wasserdampf erhitzt. \\
Der aus der Kaffeemaschine ausströmende Wasserdampf ist $T_H = 96 \, \celsius$ heiss. \\
Am Schluss haben Sie 10 \% mehr Wasser in der Tasse. (entspricht $m_D$)\\
Wie warm ist das Wasser nun?

$$ \text{Ansatz: 1. Hauptsatz} \quad \Delta Q_{\rm zu} = \Delta Q_{\rm ab} $$ 
$$ m_W \cdot c_W \, (T_M - T_K) = q_s \cdot m_D + m_D \cdot c_W \, (T_H - T_M) $$


\subsection{Eis in Wasser schmelzen}

In einem Gefäss beifinden sich $m_W = 1 \, \kilogram$ Wasser. \\
Dazu wird ein Eiswürfel von $m_E = 20 \, \gram$ gegeben. \\
Das Eis hat eine Temperatur von $T_E = -5 \, \celsius$ und das Wasser hat eine Temperatur $T_W$. Die Temperatur $T_0$ steht 
für $0 \, \celsius$ bzw. $275.15 \, \kelvin$ \\
Gesucht ist die Mischtemperatur $T_M$ 

$$ \Delta Q_{\rm ab} = \Delta Q_{\rm zu} $$
$$ m_W \cdot c_W \cdot (T_W - T_M) = m_E \cdot c_E \cdot (T_0 - T_E) + q_f \cdot m_E + m_E \cdot c_W \cdot (T_M - T_0) $$


